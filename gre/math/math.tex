% Exam Template for UMTYMP and Math Department courses
%
% Using Philip Hirschhorn's exam.cls: http://www-math.mit.edu/~psh/#ExamCls
%
% run pdflatex on a finished exam at least three times to do the grading table on front page.
%
%%%%%%%%%%%%%%%%%%%%%%%%%%%%%%%%%%%%%%%%%%%%%%%%%%%%%%%%%%%%%%%%%%%%%%%%%%%%%%%%%%%%%%%%%%%%%%

% These lines can probably stay unchanged, although you can remove the last
% two packages if you're not making pictures with tikz.
\documentclass[answers, 11pt]{exam}
\RequirePackage{amssymb, amsfonts, amsmath, latexsym, verbatim, xspace, setspace}
\RequirePackage{tikz, pgflibraryplotmarks}

% By default LaTeX uses large margins.  This doesn't work well on exams; problems
% end up in the "middle" of the page, reducing the amount of space for students
% to work on them.
\usepackage[margin=1in]{geometry}
\usepackage{color}
\usepackage{ulem}
\usepackage[T1]{fontenc} %% ensure the '>' and '<' correctly rendered

% Customize the Solution box
\renewcommand{\solutiontitle}{\noindent\textbf{Explanation:}\par\noindent}

% Here's where you edit the Class, Exam, Date, etc.
\newcommand{\class}{GRE MATH}
% \newcommand{\term}{Spring 2012}
% \newcommand{\examnum}{Exam 2}
\newcommand{\examdate}{2/8/15}
\newcommand{\timelimit}{90 Minutes}

% For an exam, single spacing is most appropriate
\singlespacing
% \onehalfspacing
% \doublespacing

% For an exam, we generally want to turn off paragraph indentation
\parindent 0ex

\begin{document} 

% These commands set up the running header on the top of the exam pages
\pagestyle{head}
\firstpageheader{}{}{}
\runningheader{\class}{Page \thepage\ of \numpages}{\examdate}
\runningheadrule

\begin{flushright}
\begin{tabular}{p{2.8in} r l}
\textbf{\class} & \\
% \textbf{\term} &&\\
% \textbf{\examnum} &&\\
\textbf{\examdate} &  \\
\textbf{Time Limit: \timelimit} & \textbf{Name (Print):} & \makebox[2in]{\hrulefill}
\end{tabular}\\
\end{flushright}
\rule[1ex]{\textwidth}{.1pt}


This question bank contains \numpages\ pages (including this cover page) and
\numquestions\ questions. 

% You may \textit{not} use your books, notes, or any calculator on this exam.\\

% You are required to show your work on each problem on this exam.  The following rules apply:\\

%\begin{minipage}[t]{3.7in}
\vspace{0pt}
\begin{itemize}

\item Questions that I got wrong during the preparation for GRE MATH

\item The questions selected from the following sources:

\begin{itemize}
\item \textbf{Magoosh}
\item \textbf{XDF Redbook}
\end{itemize}
% \item \textbf{If you use a ``fundamental theorem'' you must indicate this} and explain
% why the theorem may be applied.

% \item \textbf{Organize your work}, in a reasonably neat and coherent way, in
% the space provided. Work scattered all over the page without a clear or

% \item \textbf{Mysterious or unsupported answers will not receive full
% credit}.  A correct answer, unsupported by calculations, explanation,
% or algebraic work will receive no credit; an incorrect answer supported
% by substantially correct calculations and explanations might still receive
% partial credit.


% \item If you need more space, use the back of the pages; clearly indicate when you have done this.
\end{itemize}

% Do not write in the table to the right.
% \end{minipage}

% \hfill
% \begin{minipage}[t]{2.3in}
%% \begin{minipage}[t]{3in}
%% \framebox[\textwidth]{test out for legend!}
% \vspace{0pt}
% \cellwidth{3em}
% \gradetablestretch{2}
% \vqword{Question}
% \addpoints % required here by exam.cls, even though questions haven't started yet.	
% \gradetable[v]%[pages]  % Use [pages] to have grading table by page instead of question
% \end{minipage}

\newpage % End of cover page

%%%%%%%%%%%%%%%%%%%%%%%%%%%%%%%%%%%%%%%%%%%%%%%%%%%%%%%%%%%%%%%%%%%%%%%%%%%%%%%%%%%%%
%
% See http://www-math.mit.edu/~psh/#ExamCls for full documentation, but the questions
% below give an idea of how to write questions [with parts] and have the points
% tracked automatically on the cover page.
%
%
%%%%%%%%%%%%%%%%%%%%%%%%%%%%%%%%%%%%%%%%%%%%%%%%%%%%%%%%%%%%%%%%%%%%%%%%%%%%%%%%%%%%%

\begin{questions}

%\fullwidth{\Large \textbf{Planning}}
\section{Magoosh}
% Basic question
% \question[10] Differentiate $f(x)=x^2$ with respect to $x$.
\question[1]
The Sargon Corporation, which employs both men and women, offers an optional stock-option buy-in program to its employees. If 85\% of the men and 77\% of the women choose to participate in this plan, then which of the following could be the total number of employees? Indicate all possible values for the number of employees.
\begin{choices}
\choice 100
\CorrectChoice 200
\choice 350
\CorrectChoice 460
\choice 525
\CorrectChoice 640
\choice 750
\CorrectChoice 880
\end{choices}

\begin{solution}
Data warehouses (\textit{Correct Answer C}) are typically used to store and manage large volumes of data that is often
historical in nature and that is used primarily for analysis. Thus, a data warehouse could be used to identify sales patterns
for products sold within the past three years or to summarize sales by region, on a quarterly basis. \textit{InfoSphere Warehouse} is offered for data warehouse.
\par

Online transaction processing (OLTP) systems (\textit{Answer D}), are designed to support day-to-day, mission-critical business
activities such as web-based order entry and stock trading. \textit{DB2 pureScale feature} is offered for OLTP workloads. 
\par

{\color{red} Analytical} workloads (\textit{Answer A}) are better handled by a specialized product
known as \textit{DB2 for i and by IBM BLU Acceleration}, which is currently available
only with DB2 10.5 for LUW.

\end{solution}

\question[1]
Which product can be used to tune performance for a single query?
\begin{choices}
\CorrectChoice IBM Data Studio
\choice IBM Control Center
\choice IBM Data Administrator
\choice IBM Workload Manager
\end{choices}

\begin{solution}
\textit{IBM Data Studio} (\textit{Correct Answer A}) is an Eclipse-based, integrated
development environment (IDE) that can be used to perform instance and database
administration, routine (SQL procedure, SQL functions, etc.) and application
development, and performance-tuning tasks. It replaces the \textit{DB2 Control Center} 
(\textit{Answer B}) as the standard GUI tool for DB2 database administration and
application development.
\par

\textit{IBM Workload Manager}, or WLM (\textit{Answer D}) is a comprehensive workload
management feature that can help identify, manage, and control database workloads to
maximize database server \textcolor{red}{throughput} and \textcolor{red}{resource
 utilization}.
\par

There is \textbf{NO} such product as IBM Data Administrator (\textit{Answer C}).

\end{solution}

\newpage
\addpoints

\question[1]
Which two DB2 products are suitable for very large data warehouse applications? (Choose two.)
\begin{choices}
\choice DB2 for i
\CorrectChoice DB2 for AIX
\CorrectChoice DB2 for z/OS
\choice DB2 pureScale
\choice DB2 Express-C
\end{choices}

\begin{solution}
\textit{DB2 for i} (\textit{Answer A}), formerly known as DB2 for i5/OS, is an
advanced, 64-bit Relational Database Management System that leverages the high
performance, virtualization, and energy efficiency features of IBM's Power Systems; 
its self-managing attributes, security, and built-in analytical processing functions
make \textit{DB2 for i} an ideal database server for {\color{red} applications that 
are analytical in nature}.
\par

\textit{DB2 for z/OS} (\textit{Correct Answer C}) is a multiuser, full-function
database management system that has been designed specifically for z/OS, 
IBM's flagship mainframe operating system. Tightly integrated with the IBM mainframe,
\textit{DB2 for z/OS} leverages the strengths of System z 64-bit architecture to
provide, among other things, the ability to support complex {\color{red} data
warehouses}. 
\par

In addition to DB2 for z/OS, all of the DB2 Editions available \textbf{except} 
\textit{DB2 Express-C} (\textit{Answer E}) and \textit{DB2 Express Edition} can be
used to create data warehouse and OLTP environments. \textbf{However},
IBM offers two solutions that are tailored specifically for one workload type or
the other: \textit{InfoSphere Warehouse} for {\color{red} data warehousing} workloads
and the \textit{DB2 pureScale Feature} (\textit{Answer D}) for {\color{red} OLTP}
workloads.

\end{solution}


\question[1]
What is the DB2 Workload Manager (WLM) used for?
\begin{choices}
\choice To identify, diagnose, solve, and prevent performance problems in DB2 products and associated applications.
\CorrectChoice To customize execution environments for the purpose of controlling system resources so that one
		department or service class does not overwhelm the system.
\choice To respond to significant changes in a database's workload by dynamically distributing available memory resources
		among several different database memory consumers.
\choice To improve the performance of applications that require frequent, but relatively transient, simultaneous user
		connections by allocating host database resources only for the duration of an SQL transaction.
\end{choices}

\begin{solution}
\textit{IBM InfoSphere Optim Performance Manager Extended Edition} can be used to 
identify, diagnose, solve, and prevent performance problems in DB2 products and
associated applications (\textit{Answer A})
\par

With \textit{DB2 Workload Manager}, it is possible to customize execution
environments so that no single workload can control and consume all of the system
resources available. (This prevents any one department or service class from
overwhelming the system.) (\textit{Correct Answer B})
\par

The \textit{Self-Tuning Memory Manager (STMM)} responds to significant changes in
a database's workload by dynamically distributing available memory resources among
several different database memory consumers. (\textit{Answer C})
\par

The \textit{Connection Concentrator} improves the performance of applications that
require frequent, but relatively transient, simultaneous user connections by
allocating host database resources only for the duration of an SQL transaction. 
(\textit{Answer D})

\end{solution}


\question[1]
Which of the following is NOT a characteristic of a data warehouse?
\begin{choices}
\CorrectChoice Sub-second response time
\choice Voluminous historical data
\choice Heterogeneous data sources
\choice Summarized queries that perform aggregations and joins
\end{choices}

\begin{solution}
Sub-second response time is a feature of OLTP systems.
\end{solution}

\question[1]
Which statement about the DB2 pureScale feature is NOT true?
\begin{choices}
\choice The DB2 pureScale feature provides a database cluster solution for nonmainframe platforms.
\CorrectChoice The DB2 pureScale feature is only available as part of DB2 Advanced Enterprise Server Edition.
\choice The DB2 pureScale feature can only work with the General Parallel File System (GPFS) file system.
\choice The DB2 pureScale feature is best suited for online transaction processing (OLTP) workloads.
\end{choices}

\begin{solution}
The \textit{DB2 pureScale feature} is included as part of \textit{DB2 Workgroup
Server Edition (WSE)}, \textit{DB2 Enterprise Server Edition (ESE)}, and 
\textit{DB2 Advanced Enterprise Server Edition (AESE)}.
\end{solution}

\newpage
\addpoints

\question[1]
Which two statements about large object (LOB) locators are true? (Choose two.)
\begin{choices}
\CorrectChoice A LOB locator represents a value for a LOB resource that is stored in a database.
\CorrectChoice A LOB locator is a simple token value that is used to refer to a much bigger LOB value.
\choice A LOB locator is a special data type that is used to store LOB data in external binary files.
\choice A LOB locator represents a value for a LOB resource that is stored in an external binary file.
\choice A LOB locator is a mechanism that acts similar to an index in the way that is organizes LOB values
		so they can be quickly located in response to a query.
\end{choices}

\begin{solution}
A LOB locator is a mechanism that refers to a LOB value from within a transaction.
LOB locator is \textbf{NOT} a data type (\textit{Answer C}), nor is it a database
object. Instead, it is a token value-in the form of a host variable-that is used to
refer to a much bigger LOB value.
\par

LOB data types-\textbf{not LOB locators}-are used to store binary data values in a DB2 database
\textit{Answer D}.
\par

LOB locators \textbf{do not} store copies of LOB data (this is make it different from index)-they
store a description of a base LOB value, and the actual data that a LOB locator refers to is only
materialized when it is assigned to a specific location, such as an application host variable or
another table record (\textit{Answer E})


\end{solution}

\question[1]
Which type of database workload typically involves making changes to a small number of records within
a single transaction?
\begin{choices}
\choice Decision support
\choice Data warehousing
\choice Online analytical processing (OLAP)
\CorrectChoice Online transaction processing (OLTP)
\end{choices}

\begin{solution}
An online transaction processing (OLTP) environment often consists of hundreds to thousands of
users issuing millions of transactions per day against databases that vary in size.
Consequently, the volume of data affected may be very large, even though {\color{red} each transaction
typically makes changes to only a small number of records}.
\par

Data warehousing (\textit{Answer B}) involves storing and managing large volumes of data that is often 
historical in nature and that is used primarily for analysis. Consequently, data warehouses are frequently
used in reporting, online analytical processing (OLAP) (\textit{Answer C}), and decision support 
(\textit{Answer A}) environments.


\end{solution}


\question[1]
Which of the following is NOT a characteristic of an OLTP database?
\begin{choices}
\choice Current data
\choice Frequent updates
\choice Granular transactions
\CorrectChoice Optimized for queries
\end{choices}

\begin{solution}
Data warehouse workloads typically consist of:
\begin{itemize}
\item bulk load operations
\item short-running queries
\item long-running complex queries
\item random queries
\item occasional updates to data
\item execution of online utilities
\end{itemize}
Therefore, data warehouses are optimized for queries (\textit{Correct Answer D}).
\par

Online transaction processing (OLTP) systems features:
\begin{itemize}
\item Support day-to-day, mission-critical business activities
\item Support hundreds to thousands of users issuing millions of transactions per day (\textit{Answer C})
against databases that vary in size
\item Response time requirements tend to be subsecond
\item Workloads tend to be a mix of real-time insert, update (\textit{Answer B}), and delete
operations against current-as opposed to historical-data (\textit{Answer A})
\end{itemize}

\end{solution}

\question[1]
Which two platforms support DB2 10.1 pureScale environments? (Choose two.)
\begin{choices}
\choice IBM mainframes running z/OS
\CorrectChoice IBM p Series servers running AIX
\choice IBM p Series servers running Linux
\CorrectChoice IBM x Series servers running Linux
\choice IBM x Series servers running a supported version of Windows
\end{choices}

\begin{solution}
\textit{DB2 pureScale (Version 10.1)} can \textbf{ONLY} be installed on IBM p Series or x Series
servers that are running either the AIX (p Series) or the Linux (x Series) operating system.
\par

\textit{DB2 pureScale} \textbf{CANNOT} be installed on IBM mainframes running z/OS (\textit{Answer A}),
IBM p Series servers running Linux (\textit{Answer C}), or IBM x Series servers running Windows
(\textit{Answer E}).

\end{solution}
% \newpage
% \addpoints

\question[1]
Which tool can analyze and provide recommendations for tuning individual queries?
\begin{choices}
\choice IBM InfoSphere Data Architect
\CorrectChoice IBM InfoSphere Optim Query Tuner
\choice IBM InfoSphere Optim pureQuery Runtime
\choice IBM InfoSphere Optim Performance Manager Extended Edition
\end{choices}

\begin{solution}
\textit{IBM InfoSphere Data Architect} offers a complete solution for designing, modeling,
discovering, relating, and standardizing data assets (\textit{Answer A}).
\par

\textit{IBM InfoSphere Optim Query Tuner}, often referred to as the Query Tuner, can analyze
and make recommendations on ways to tune existing queries, as well as provide expert advice
on writing new queries (\textit{Correct Answer B}).
\par

\textit{IBM InfoSphere Optim pureQuery Runtime} bridges the gap between data and Java technology
by harnessing the power of SQL within an easy-to-use Java data access platform (\textit{Answer C}).
\par

\textit{IBM InfoSphere Optim Performance Manager Extended Edition} can identify, diagnose, solve, 
and prevent performance problems in DB2 products and associated applications (\textit{Answer D}).


\end{solution}

\question[1]
Which SQL statement will create a table named EMPLOYEE that can be used to store XML data?
\begin{choices}
\CorrectChoice CREATE TABLE employee (empid INT, resume XML)
\choice CREATE TABLE employee (empid INT, resume XML(2000))
\choice CREATE TABLE employee (empid INT, resume CLOB AS XML)
\choice CREATE TABLE employee (empid INT, resume CLOB USING XML)
\end{choices}

\begin{solution}
The \textit{DB2 pureXML} offers a simple and efficient way to create a "hybrid" DB2 database
that allows XML data to be stored in its native, hierarchical format. With \textit{pureXML}, XML
documents are stored in tables that contain one or more columns that have been defined with
the XML data type.
\par

Since the XML data type does not require a size specification (\textit{Answer B}), and because
"CLOB AS XML" (\textit{Answer C}) and "CLOB USING XML" (\textit{Answer D}) are not valid column
definitions, the only CREATE TABLE statement shown that will execute successfully is:

\begin{verbatim}
CREATE TABLE employee (empid INT, resume XML)
\end{verbatim}


\end{solution}


\question[1]
What DB2 product provides a complete data warehousing solution that contains components
that facilitate data warehouse construction and administration?
\begin{choices}
\choice DB2 pureScale Feature
\choice DB2 Workload Manager
\CorrectChoice IBM InfoSphere Warehouse
\choice Database Partitioning Feature
\end{choices}

\begin{solution}
\textit{DB2 pureScale Feature} enables a DB2 for LUW database to continuously process incoming requests,
even if multiple system componets fail simultaneously, which makes it ideal for OLTP workloads where 
high availability is crucial (\textit{Answer A}).
\par

\textit{DB2 Workload Manager (WLM)} is a comprehensive workload management feature that can help identify,
manage, and control database workloads to maximize database server throughput and resource utilization
(\textit{Answer B}).
\par

\textit{IBM InfoSphere Warehouse} is a complete data warehousing solution that contains components
that facilitate data warehouse construction and administration, as well as tools that enable
embedded data mining and multidimensional online analytical processing (OLAP) (\textit{Correct Answer C}).
\par

\textit{Data Partitioning Feature (DPF)} provides the ability to divide very large databases into 
multiple parts (known as partitions) and store them across a cluster of inexpensive servers 
(\textit{Answer D}).

\end{solution}

\question[1]
Which statement about IBM Data Studio is NOT true?
\begin{choices}
\CorrectChoice The IBM Data Studio administration client can be installed on servers running 
Red Hat Linux, SUSE Linux, Windows, and AIX.
\choice IBM Data Studio replaces the DB2 Control Center as the standard GUI interface for DB2 
database administration and application development.
\choice IBM Data Studio is an Eclipsed-based, integrated development environment (IDE) that can be 
used to perform instance and database administration.
\choice IBM Data Studio allows users to connect to a DB2 database using a wizard; however, users 
are required to provide login credentials before a connection will be established.
\end{choices}

\begin{solution}
There are three different IBM Data Studio components to choose from: \textit{IBM Data Studio
administration client}, \textit{IBM Data Studio full client}, and \textit{IBM Data Studio web console}.
All three components can be installed on servers running Red Hat Linux, SUSE Linux, and Windows; 
IBM Data Studio web console can be installed on servers running the AIX operating system as well.
(\textbf{IBM Data Studio administration client cannot be installed on AIX servers}).

\end{solution}

\question[1]
Which statement about inline large objects (LOBs) is NOT true?
\begin{choices} 
\choice When a table contains columns with inline LOBs, fewer rows can fit on a page.
\choice Inline LOBs are created by appending the INLINE LENGTH clause to a LOB column's definition.
\CorrectChoice Because DML operations against inline LOBs are never logged, their use can reduce logging overhead.
\choice Inline LOBs improve query performance by storing LOB data in the same data pages as the rest of a
table's rows, rather than in a separate LOB storage object.
\end{choices}

\begin{solution}
By default, Large object (LOB) data is stored in separate LOB storage objects and changes to LOB data
are not recorded in transaction log files. However, LOBs that are relatively small can be stored in the
same data pages as the rest of a table's rows-such LOBs are referred to as \textit{inline LOBs}, and
transactions that modify inline LOB data are always logged. Consequently, the use of inline LOBs can
increase-\textbf{not reduce}-logging overhead.
\par

Inline LOBs are created by appending the INLINE LENGTH clause to a LOB column's definition (\textit{Answer B}), which can be specified via the \texttt{CREATE TABLE} or \texttt{ALTER TABLE} statement. Inline LOBs
improve the performance of queries that access LOB data since no additional I/O is needed to access
this type of data (\textit{Answer D}). However, when a table has columns with inline LOBs in it,
fewer rows will fit on a page (\textit{Answer A}).

\end{solution}

\newpage
\addpoints
%\fullwidth{\Large \textbf{Security}}
\section{Security}
\question[1]
{\color{red} A user named USER1 has been granted \texttt{DATAACCESS} authority for a database named PAYROLL. What is user USER1 allowed to do?}
\begin{choices}
	\choice Implicitly create a new schema in the PAYROLL database.
	\choice Grant and revoke priviledges on objects that reside in the PAYROLL database.
	\CorrectChoice Retrieve and change data stored in user tables, views. and materialized query tables.
	\choice Create database objects, issue database-specific DB2 commands, and run DB2 utilities that 
	do not change data.
\end{choices}

\begin{solution}
\begin{itemize}
\item \textit{Answer A} IMPLICIT\_SCHEMA database privilege
\item \textit{Answer B} ACCESSCTRL
\item \textit{Answer D} DBMAINT
\end{itemize}
\end{solution}

\question[1]
Which attribute is NOT needed to define a trusted context?
\begin{choices}
\choice A system authorization ID
\choice A data stream encryption value
\CorrectChoice A system authorization password
\choice The IP address or domain name of an incoming connection
\end{choices}

\question[1]
Which SQL statement will take the ability to run an Embedded SQL application named 
PERF\textunderscore REVIEW that calls a package named CORP.CALC\textunderscore Bonus away from a user named USER1?
\begin{choices}
\choice \texttt{REVOKE EXECUTION ON APPLICATION} perf\textunderscore review \texttt{FROM} user1
\CorrectChoice \texttt{REVOKE EXECUTION ON PACKAGE} corp.calc\textunderscore bonus \texttt{FROM} user1
\choice \texttt{REVOKE EXECUTION ON APPLICATION} perf\textunderscore review \texttt{PACKAGE} corp.calc
\textunderscore bonus \texttt{FROM} user1
\choice \texttt{REVOKE EXECUTION ON APPLICATION} perf\textunderscore review \texttt{USING PACKAGE} corp.calc\textunderscore bonus \texttt{FROM} user1
\end{choices}

\begin{solution}
No APPLICATION database objects (or related privileges)
\end{solution}

\newpage

\question[1]
{\color{red}If a user is given \texttt{SELECT} privilege on a table named EMPLOYEES, which two actions are 
they allowed to perform? (Choose two.)}
\begin{choices}
\choice Add data to the EMPLOYEE table.
\CorrectChoice Create a view on the EMPLOYEE table.
\CorrectChoice Retrieve data from the EMPLOYEE table.
\choice Create an index for the EMPLOYEE table.
\choice Change the definition for the EMPLOYEE table.
\end{choices}

\begin{solution}
\begin{itemize}
\item \textit{Answer A} INSERT table privilege
\item \textit{Answer D} INDEX table privilege
\item \textit{Answer E} ALTER table privilege
\end{itemize}
\end{solution}

\question[1]
Which SQL statement will allow a user named USER1 to both remove records from a table named
SALES and give the ability to remove records from the SALES table to others?
\begin{choices}
\CorrectChoice \texttt{GRANT DELETE ON TABLE} sales \texttt{TO} user1 \texttt{WITH GRANT OPTION}
\choice \texttt{GRANT REMOVE ON TABLE} sales \texttt{TO} user1 \texttt{WITH GRANT OPTION}
\choice \texttt{GRANT DELETE ON TABLE} sales \texttt{TO} user1 \texttt{WITH GRANT PRIVILEGES}
\choice \texttt{GRANT REMOVE ON TABLE} sales \texttt{TO} user1 \texttt{WITH GRANT PRIVILEGES}
\end{choices}

\begin{solution}
No "REMOVE" table privilege and "WITH GRANT PRIVILEGES" clause
\end{solution}

\question[1]
{\color{red} If a user is granted the \texttt{BIND} privilege, what are they allowed to do?}
\begin{choices}
\choice Create a new package.
\CorrectChoice Bind or rebind (recreate) a specific package.
\choice Register user-defined functions (UDFs) and procedures.
\choice Associate user-defined functions (UDFs) and procedures with specific database objects.
\end{choices}

\begin{solution}
\begin{itemize}
\item \textit{Answer A} BINDADD table privilege
\item \textit{Answer C} CREATE\_EXTERNAL\_ROUTINE table privilege
\item \textit{Answer D} No such authority or privilege exists
\end{itemize}
\end{solution}

\question[1]
{\color{red} Which statement about \texttt{Security Administrator (SECADM)} authority is true?}
\begin{choices}
\choice Users with \texttt{SECADM} authority are not allowed to access data stored in system 
catalog tables and views.
\CorrectChoice Only users with \texttt{SECADM} authority are allowed to grant and revoke \texttt{SECADM}
authority to/from others.
\choice When a user with \texttt{SECADM} authority creates a database, that user is automatically granted
\texttt{DBADM} authority for that database.
\choice With DB2 for z/OS, \texttt{SYSADM} authority and \texttt{SECADM} authority are combined under
\texttt{SYSADM} authority and cannot be separated.
\end{choices}

\begin{solution}
\begin{itemize}
\item \textit{Answer A} CAN access data stored in system catalog tables and views; CANNOT access user data
\item \textit{Answer C} Individuals who possess SECADM CANNOT create databases
\item \textit{Answer D} CAN separate: set the SEPARATE\_SECURITY system parameter on panel DSNTIPP1 to YES
during installation or migration
\end{itemize}
\end{solution}

\question[1]
{\color{red} Which statement about trusted context is true?}
\begin{choices}
\choice Trusted context objects can only be defined by someone with \texttt{SYSADM} or \texttt{SECADM}
authority.
\choice An authorization ID, IP address, encryption value, and authentication type must be identified
before a trusted context can be defined.
\CorrectChoice After a trusted connection is established, if a switch request is made with an authorization ID
that is not allowed on the connection, the connection is placed in the "Unconnected" state.
\choice If a trusted context is assigned to a role, any authorization ID that uses the trusted context
will acquire the authorities and privileges that have been assigned to the role; any authorities or
privileges that have been granted to the authorization ID are ignored.
\end{choices}

\begin{solution}
\begin{itemize}
\item \textit{Answer A} can ONLY be defined by SECADM
\item \textit{Answer B} An authentication type is NOT needed to define a trusted context
\item \textit{Answer D} any authorities or privileges that have been granted to the authorization ID are 
NOT ignored. (DB2 z/OS extended trusted context)
\end{itemize}
\end{solution}

\newpage
\question[1]
{\color{red} If a user has \texttt{ACCESSCTRL} authority, which two authorities and/or privileges are they 
allowed to grant to others? (Choose two.)}
\begin{choices}
\choice \texttt{SYSADM}
\choice \texttt{SECADM}
\CorrectChoice \texttt{EXECUTE}
\CorrectChoice \texttt{CREATETAB}
\choice \texttt{ACCESSCTRL}
\end{choices}

\begin{solution}
SYSADM (\textit{Answer A}), SECADM (\textit{Answer B}), and ACCESSCTRL (\textit{Answer D}) are system-
level and database-level authorities-not object privilege. Consequently, they can only be granted by 
SECADM.
\end{solution}

\question[1]
Which of the following is used to group a collection of privileges together so that they can be
simultaneously granted to and revoked from multiple users?
\begin{choices}
\CorrectChoice Role
\choice Catalog
\choice Function
\choice Collection
\end{choices}

\question[1]
{\color{red}Which method for restricting data access relies on the server or the local DB2 subsystem to 
prevent unauthorized users from accessing data stored in a database?}
\begin{choices}
\choice Privileges
\CorrectChoice Authentication
\choice Label-based access control
\choice Row and column access control
\end{choices}

\begin{solution}
Server is the key word in this question. 
\end{solution}

\question[1]
When is an SQL search condition used to limit access to data in a table?
\begin{choices}
\choice When \texttt{mandatory access control (MAC)} is used to protect the table.
\choice When \texttt{label-based access control (LBAC)} is used to protect the table.
\choice When \texttt{discretionary access control (DAC)} is used to protect the table.
\CorrectChoice When \texttt{row and column access control (RCAC)} is used to protect the table.
\end{choices}

\question[1]
{\color{red} Which SQL statement will give user USER1 the ability to create tables in a table space named USERSPACE2?}
\begin{choices}
\CorrectChoice \texttt{GRANT USE OF TABLESPACE} userspace2 \texttt{TO} user1
\choice \texttt{GRANT ALTER ON TABLESPACE} userspace2 \texttt{TO} user1
\choice \texttt{GRANT USAGE OF TABLESPACE} userspace2 \texttt{TO} user1
\choice \texttt{GRANT CREATETAB ON TABLESPACE} userspace2 \texttt{TO} user1
\end{choices}

\begin{solution}
Only one table space privilege exists-the USE (or USE OF TABLESPACE) privilege
\end{solution}

\question[1]
Which SQL statement will give user USER1 the ability to assign a comment to a table named MYTABLE?
\begin{choices}
\CorrectChoice \texttt{GRANT ALTER ON TABLE} mytable \texttt{TO} user1
\choice \texttt{GRANT USAGE ON TABLE} mytable \texttt{TO} user1
\choice \texttt{GRANT INSERT ON TABLE} mytable \texttt{TO} user1
\choice \texttt{GRANT UPDATE ON TABLE} mytable \texttt{TO} user1
\end{choices}

\question[1]
Which privileges are needed to invoke an SQL stored procedure that queries a table?
\begin{choices}
\choice \texttt{CALL} privilege on the procedure; \texttt{SELECT} privilege on the table.
\CorrectChoice \texttt{EXECUTE} privilege on the procedure; \texttt{SELECT} privilege on the table.
\choice \texttt{CALL} privilege on the procedure; \texttt{REFERENCES} privilege on the table.
\choice \texttt{EXECUTE} privilege on the procedure; \texttt{REFERENCES} privilege on the table.
\end{choices}

\question[1]
Which privileges allows a user to use the \texttt{PREVIOUS VALUE} and \texttt{NEXT VALUE} sequence 
expressions?
\begin{choices}
\choice \texttt{USE}
\choice \texttt{ALTER}
\CorrectChoice \texttt{USAGE}
\choice \texttt{EXECUTE}
\end{choices}

\begin{solution}
sequence only has two privileges - USAGE (\textbf{NOT} use) \& ALTER. ALTER privilege allows  user to
\begin{itemize}
\item restarting the sequence
\item changing the increment value for the sequence
\item add or change the comment associated with a sequence
\end{itemize}
\end{solution}

\question[1]
A table named CUSTOMER was created as follows:
\begin{verbatim}
CREATE TABLE customer
(cust_id   INTEGER NOT NULL PRIMARY KEY,
 f_name    VARCHAR(30),
 l_name    VARCHAR(40),
 cc_number NUMERIC(16,0) NOT NULL)
\end{verbatim}
Which two actions will prevent unauthorized users from accessing credit card number (CC\textunderscore 
NUMBER) information? (Choose two.)
\begin{choices}
\choice Assign the CC\textunderscore NUMBER column to a restricted role that only authorized users
are allowed to access.
\choice Only grant \texttt{ACCESSCTRL} authority for the CC\textunderscore NUMBER column to users who
need to access credit card number information.
\choice Alter the table definition so that CC\textunderscore NUMBER data is stored in a separate schema
that only authorized users are allowed to access.
\CorrectChoice Create a view for the CUSTOMER table that does not contain the CC\textunderscore NUMBER column
and require unauthorized users to use the view.
\CorrectChoice Create a column mask for the CC\textunderscore NUMBER column with the ENABLE option specified and 
alter the CUSTOMER table to activate column access control.
\end{choices}

\begin{solution}
\begin{itemize}
\item \textit{Answer A} No such thing as a "restricted" role
\item \textit{Answer B} ACCESSCTRL does not have ability to retrieve data
\item \textit{Answer C} Objects such as tables, views, and index can be stored in different schemas, but 
individual table columns cannot. And even if they could, there is no privilege that can be used to 
prevent certain individuals from accessing objects that have been stored in a particular schema
\end{itemize}
\end{solution}

\question[1]
{\color{red} Which authority is needed to create and drop databases?}
\begin{choices}
\choice \texttt{DBADM}
\choice \texttt{DBCTRL}
\CorrectChoice \texttt{SYSCTRL}
\choice \texttt{SYSMAINT}
\end{choices}

\newpage

\question[1]
Which statement regarding \texttt{label-based access control (LBAC)} is true?
\begin{choices}
\choice Two types of security label components are supported: array and tree.
\CorrectChoice Every LBAC-protected table must have only one security policy associated with it.
\choice To configure a table for row-level LBAC protect, you must include the \texttt{SECURED WITH}
clause with each column's definition.
\choice To configure a table for column-level LBAC protection, you must include a column with the
\texttt{DB2SECURITYLABEL} data type in the table's definition.
\end{choices}

\begin{solution}
\begin{itemize}
\item \textit{Answer A} Three types of security label components are supported: SET, ARRAY and TREE
\item \textit{Answer C} To configure a table for row-level LBAC protect, you must associate a security 
policy with the table (using the SECURITY POLICY clause of the CREATE TABLE or ALTER TABLE statement) and
\textit{include a column with the data type in the table's definition}
\item \textit{Answer D} To configure a table for column-level LBAC protect, you must associate a security
policy with the table and \textit{configure each of the table's columns for protection by adding the
SECURED WITH clause to every column's definition}
\item \textit{Answer C} and \textit{D} are incorrect because they state just the opposite
\end{itemize}
\end{solution}

\question[1]
Which method for restricting data access relies on an SQL CASE expression to control the conditions
under which a user can access for a column?
\begin{choices}
\choice Authority
\choice Authentication
\choice Label-based access control
\CorrectChoice Row and column access control
\end{choices}

\question[1]
Which two statements about Row and column Access Control (RCAC) are valid? (Choose two.)
\begin{choices}
\choice A column mask's access control rule is defined by an SQL search condition.
\CorrectChoice A column mask's access control rule is defined by an SQL CASE expression.
\CorrectChoice A row permission's access control rule is defined by an SQL search condition.
\choice A row permission's access control rule is defined by an SQL CASE expression.
\choice A column mask's access control rule is defined by a \texttt{SECURED WITH} clause of 
a \texttt{CREATE TABLE} or \texttt{ALTER TABLE} statement.
\end{choices}

\newpage

\question[1]
Which privilege is needed to invoke a stored procedure?
\begin{choices}
\choice \texttt{USE}
\choice \texttt{CALL}
\choice \texttt{USAGE}
\CorrectChoice \texttt{EXECUTE}
\end{choices}

\newpage
%\fullwidth{\Large \textbf{Working with Databases and Database Objects}}
\section{Working with Databases and Database Objects}
\question[1]
Which statement about views is NOT true?
\begin{choices}
	\CorrectChoice A view can be defined as being updatable or read-only.
	\choice Views obtain their data from the table(s) or view(s) they are based on.
	\choice A view can be used to limit a user's ability to retrieve data from a table
	\choice The SQL statement provided as part of a view's definition determines what data is 
	presented when the view is referenced.
\end{choices}

\begin{solution}
	views can be defined as being \textit{insertable}, updatabale, \textit{deletable}, or read-only.
\end{solution}

\question[1]
{\color{red}If the following SQL statement is executed:
\begin{verbatim}
CREATE DISTINCT TYPE pound_sterling AS DECIMAL (9,2) WITH COMPARISONS
\end{verbatim}
Which event will NOT happen?}
\begin{choices}
	\choice A user-defined data type that can be used to store numerical data as British currency will be
	created.
	\choice Six comparison functions will be created so that POUND\_STERLING values can be compared to each
	other.
	\choice Two casting functions will be created so that POUND\_STERLING values can be converted to DECIMAL
	values, and vice versa.
	\CorrectChoice A compatibility function will be created so all of DB2's built-in functions that accept DECIMAL
	values as input can be used with POUND\_STERLING data.
\end{choices}

\begin{solution}
	Distinct types cannot be used as arguments for most built-in functions, and built-in data types cannot be used in arguments
	or operands that expect distinct data types. Instead, user-defined functions (UDFs) that provide similar functionality must 
	be developed if that capability is needed.
\end{solution}

\question[1]
{\color{red}If the following SQL statements are executed:}
\begin{verbatim}
CREATE TABLE sales(
order_num      INTEGER NOT NULL,
customer_name  VARCHAR(50),
amount_due     DECIMAL(6,2));
CREATE UNIQUE INDEX idx_ordernum ON sales(order_num);
\end{verbatim}
{\color{red}Which two statements are true? (Choose two.)}
\begin{choices}
	\CorrectChoice Every ORDER\_NUM value must be unique.
	\choice Duplicate ORDER\_NUM values are allowed.
	\choice No other indexes can be created for the SALES table.
	\CorrectChoice A query will return rows from the SALES table in no specific order.
	\choice Index IDX\_ORDERNUM will serve as the primary key for the SALES table.
\end{choices}

\begin{solution}
	\begin{itemize}
		\item \textit{Answer D} Because index do not physically change the order of records in a table, queries that do not take advantage of index
		will return rows from a table in no specific order (To be more concise, \textbf{cluster} clause not specified in \texttt{CREATE UNIQUE INDEX} statement)
		\item \textit{Answer E} Because ORDER\_NUM column was not defined as a primary key in the scenario presented, the index produced by the CREATE INDEX
		statement shown will \textit{not} serve as the primary key index for the SALES table
	\end{itemize}
\end{solution}

\question[1]
What is the minimum product that is needed to give applications running on personal computers the 
ability to work with DB2 databases that reside on System z platforms, without using a gateway?
\begin{choices}
	\CorrectChoice DB2 Connect Personal Edition
	\choice DB2 Connect Enterprise Edition
	\choice IBM DB2 Connect Unlimited Advanced Edition for System z
	\choice IBM DB2 Connect Unlimited Advanced Edition for System i
\end{choices}

\newpage
\question[1]
Which action does NOT need to be performed to complete the definition of an application-period temporal
table?
\begin{choices}
	\choice A business-time-begin column must be created for the table.
	\choice A business-time-end column must be created for the table.
	\choice A BUSINESS\_TIME period must be specified in a CREATE or ALTER of the table.
	\CorrectChoice A unique index must be created that prevents overlapping of the BUSINESS\_TIME period of the 
	table.
\end{choices}

\question[1]
What are buffer pools used for?
\begin{choices}
	\CorrectChoice To cache table and index data as it is read from disk.
	\choice To keep track of changes that are made to a database as they occur.
	\choice To control the amount of processor resources that SQL statements can consume.
	\choice To provide a layer of indirection between a data object and the storage where that object's data
	resides.
\end{choices}

\question[1]
{\color{red}Which statement regarding distributed requests is NOT true?}
\begin{choices}
	\CorrectChoice To implement distributed request functionality, all you need is a federated database and one
	or more remote data sources.
	\choice Distributed request functionality allows a UNION operation to be performed between a DB2 table and
	an Oracle view.
	\choice Distributed request functionality allows SQL operations to reference two or more databases or 
	relational database management systems in a single statement.
	\choice DB2 Connect provides the ability to perform distributed requests across members of the DB2 Family,
	as well as across other relational database management systems.
\end{choices}

\begin{solution}
	\begin{itemize}
		\item \textit{Answer A} To implement distributed request functionality, you need \textit{a DB2 Connect instance},
		a database that will serve as a federated database, and one or more remote data sources.
	\end{itemize}
\end{solution}

\question[1]
Which statement about indexes is NOT true?
\begin{choices}
	\choice An index can be used to enforce the uniqueness of records in a table.
	\choice Indexes provide a fast, efficient method for locating specific rows in a table.
	\choice When an index is created, metadata for the index is stored in the system catalog.
	\CorrectChoice Indexes automatically provide both a logical and physical ordering of the rows in a table.
\end{choices}

\begin{solution}
	An index is an object that contains pointers to rows in a table that are \textit{logically} ordered according to the values
	of one or more columns (known as keys). Special index known as \textit{clustering} index can cause the rows of a table to be physically
	arranged according to the ordering of their key column values, but such index are not created automatically. 
\end{solution}

\question[1]
What are Materialized Query Tables (MQTs) used for?
\begin{choices}
	\choice To physically cluster data on more than one dimension, simultaneously.
	\CorrectChoice To improve the execution performance of qualified SELECT statements.
	\choice To hold nonpresistent data temporarily, on behalf of a single application.
	\choice To track effective dates for data that is subject to changing business conditions.
\end{choices}

\begin{solution}
	\begin{itemize}
		\item \textit{Answer A} Multidimensional clustering tables (MDC)
		\item \textit{Answer C} Declared global temporary tables
		\item \textit{Answer D} application-period temporal tables
	\end{itemize}
\end{solution}


\question[1]
Which two actions must be performed to track changes made to a system-period temporal table over time?
(Choose two.)
\begin{choices}
	\CorrectChoice A history table must be created with columns that are identical to those of the system-period
	temporal table.
	\CorrectChoice The system-period temporal table must be altered using the ADD VERSIONING clause to relate it to
	a history table.
	\choice A primary key must be defined for the system-period temporal table that prevents overlapping
	of SYSTEM\_TIME periods.
	\choice A unique index must be defined on the transaction-start-id column of both the system-period 
	temporal table and its associated history table.
	\choice The system-period temporal table must be altered to add system-time-begin, system-time-end, 
	transaction-start-id, and transaction-end-id columns.
\end{choices}

\begin{solution}
	\begin{itemize}
		\item A primary key (\textit{Answer C}) or unique index (\textit{Answer D}) does not have to be included in the definition provided
		for a system-period temporal table.
		\item \textit{Answer E} transaction-end-id column is not needed (nor recognized by DB2)
	\end{itemize}
\end{solution}

\question[1]
Which database object can be used to automatically generate a numeric value that is not tied to any
specific column or table?
\begin{choices}
	\choice Alias
	\choice Schema
	\choice Package
	\CorrectChoice Sequence
\end{choices}

\newpage
\question[1]
Which column is NOT required as part of the table definition for a system-period temporal table?
\begin{choices}
	\choice A row-begin column with a TIMESTAMP(12) data type
	\choice A row-end column with a TIMESTAMP(12) data type
	\choice A transaction-start-id column with a TIMESTAMP(12) data type
	\CorrectChoice A transaction-stop-id column with a TIMESTAMP(12) data type
\end{choices}

\question[1]
Which object can NOT be enabled for compression?
\begin{choices}
	\CorrectChoice Views
	\choice Indexes
	\choice Base tables
	\choice Temporary tables
\end{choices}

\begin{solution}
	Since views do not contain data, they cannot be enabled for compression.
\end{solution}

\question[1]
What is a schema used for?
\begin{choices}
	\choice To provide an alternate name for a table or view.
	\CorrectChoice To provide a logical grouping of database objects.
	\choice To generate a series of numbers, in ascending or descending order.
	\choice To provide an alternative way of describing data stored in one or more tables.
\end{choices}

\begin{solution}
	\begin{itemize}
		\item \textit{Answer A} Alias
		\item \textit{Answer C} Sequence
		\item \textit{Answer D} View
	\end{itemize}
\end{solution}

\question[1]
Which view definition type is NOT supported?
\begin{choices}
	\choice Insertable
	\choice Updatable
	\choice Read-only
	\CorrectChoice Write-only
\end{choices}

\newpage
\question[1]
When should an application-period temporal table be used?
\begin{choices}
	\choice When you want to keep track of historical versions of a table's rows.
	\CorrectChoice When you want to define specific time periods in which data is valid.
	\choice When you want to cluster data according to the time in which rows are inserted.
	\choice When you want to cluster data on more than one key or dimension, simultaneously.
\end{choices}

\begin{solution}
	\begin{itemize}
		\item \textit{Answer A} system-period temporal table
		\item \textit{Answer C} insert time clustering (ITC) tables
		\item \textit{Answer D} multidimensional clustering (MDC) tables 
	\end{itemize}
\end{solution}

\question[1]
Which statement about buffer pools is NOT true?
\begin{choices}
	\choice Every table space must have a buffer pool assigned to it.
	\choice One buffer pool is created automatically as part of the database creation process.
	\CorrectChoice Dirty pages are automatically removed from a buffer pool when they are written to storage.
	\choice Once a page has been copied to a buffer pool, it remains there until the space it occupies is 
	needed.
\end{choices}

\question[1]
Which DB2 object can a view NOT be derived from?
\begin{choices}
	\choice Alias
	\choice View
	\choice Table
	\CorrectChoice Procedure
\end{choices}

\question[1]
Which two expressions can be used with a sequence? (Choose two.)
\begin{choices}
	\CorrectChoice NEXT VALUE
	\choice PRIOR VALUE
	\choice CURRENT VALUE
	\CorrectChoice PREVIOUS VALUE
	\choice SUBSEQUENT VALUE
\end{choices}

\newpage
\question[1]
{\color{red}Which object is a distinct data type defined into?}
\begin{choices}
	\CorrectChoice Schema
	\choice Package
	\choice Database
	\choice Table space
\end{choices}

\begin{solution}
	When table spaces, tables, index, \textit{distnct data types}, functions, stored procedures, and triggers are created, they are
	automatically assigned to (or defined into) a schema, based upon the qualifier that was provided as part of the user-supplied name.
\end{solution}

\question[1]
Which two objects can NOT be created in DB2? (Choose two.)
\begin{choices}
	\CorrectChoice Plan
	\choice Trigger
	\CorrectChoice Scheme
	\choice Function
	\choice Sequence
\end{choices}

\begin{solution}
	\begin{itemize}
		\item \textit{Answer A} While packages can contain data access plans that were implicitly created, there is no way to explicitly create a 
		data access plan.
		\item \textit{Answer C} There is no such thing as a "scheme" object, scheme objects cannot be explicitly created.
	\end{itemize}
\end{solution}

\question[1]
{\color{red}Which statement about Type 2 connections is true?}
\begin{choices}
	\choice Type 2 connections cannot be used with DB2 for z/OS
	\choice Type 2 connections are used by default with DB2 for Linux, Unix, and Windows.
	\choice Type 2 connections allow applications to be connected to only one database at a time.
	\CorrectChoice Type 2 connections allow applications to connect to and work with multiple DB2 databases 
	simultaneously.
\end{choices}

\begin{solution}
	\begin{itemize}
		\item \textit{Answer A} Type 2 connections can be used with DB2 for z/OS
		\item \textit{Answer B} Type 1 connections are used by default with DB2 LUW; Type 2 connections are used by default with DB2 z/OS.
		\item \textit{Answer C} Type 2 connections allow a single transaction to connect to and work with multiple databases simultaneously.
	\end{itemize}
\end{solution}

\question[1]
Which two statements about bitemporal tables are valid? (Choose two.)
\begin{choices}
	\choice Bitemporal tables are system tables and can only be queried by the table owner.
	\CorrectChoice When data in a bitemporal table is updated, a row is added to its associated history table.
	\choice Creating a bitemporal table is similar to creating a base table except users must define a 
	SYSTEM\_TIME\_PERIOD column.
	\CorrectChoice When querying a bitemporal table, you have the option of providing a system time-period 
	specification, a business time-period specification, or both.
	\choice Bitemporal tables must contain bitemporal-time-begin, bitemporal-time-end, and transaction-start-i
	id columns, along with SYSTEM\_TIME and BUSINESS\_TIME periods.
\end{choices}

\begin{solution}
	\begin{itemize}
		\item \textit{Answer A} Bitemporal tables are user tables, NOT system tables.
		\item \textit{Answer C} Created by executing a CREATE TABLE statement with both the PERIOD SYSTEM\_TIME clause and the PERIOD BUSINESS\_TIME clause specified.
		\item \textit{Answer D} No "bitemporal-time-begin" and "bitemporal-time-end" columns exist.
	\end{itemize}
\end{solution}

\question[1]
{\color{red}Which programming interface is widely used for database access because it allows applications to run, 
unchanged, on most hardware platforms?}
\begin{choices}
	\choice ODBC
	\choice SQLJ
	\CorrectChoice JDBC
	\choice OLE DB
\end{choices}

\question[1]
Which two types of temporal tables can be used to store time-sensitive data? (Choose two.)
\begin{choices}
	\CorrectChoice Bitemporal
	\choice Time-period
	\choice System-period
	\choice Business-period
	\CorrectChoice Application-period
\end{choices}

\newpage
\question[1]
{\color{red}In which of the following scenarios would a stored procedure be beneficial?}
\begin{choices}
	\choice An application running on a remote client needs to track every modification made to a 
	table that contains sensitive data.
	\choice An application running on a remote client needs to be able to convert degress Celsius to degrees
	Fahrenheit and vice versa.
	\choice An application running on a remote client needs to ensure that every new employee that joins the 
	company is assigned a unique, sequential employee number.
	\CorrectChoice An application running on a remote client needs to collect input values from a user, 
	perform a calculation using the values provided, and store the input data, along with the calculation
	results, in a base table.
\end{choices}

\begin{solution}
	\begin{itemize}
		\item \textit{Answer A} trigger
		\item \textit{Answer B} UDF
		\item \textit{Answer C} sequence
	\end{itemize}
\end{solution}

\question[1]
Given the following SQL statement:
\begin{verbatim}
CREATE ALIAS emp_info FOR employees
\end{verbatim}
Which two objects can the name EMPLOYEES refer to? (Choose two.)
\begin{choices}
	\CorrectChoice A view
	\CorrectChoice An alias
	\choice An index
	\choice A sequence
	\choice A procedure
\end{choices}

\question[1]
{\color{red}Which operation can NOT be performed by executing an ALTER SEQUENCE statement?}
\begin{choices}
	\CorrectChoice Change a sequence's data type.
	\choice Change whether a sequence cycles.
	\choice Establish new minimum and maximum sequence values.
	\choice Change the number of sequence numbers that are cached.
\end{choices}

\question[1]
Which object must exist before an index can be created?
\begin{choices}
	\choice View
	\CorrectChoice Table
	\choice Schema
	\choice Sequence
\end{choices}

\question[1]
{\color{red}If the following SQL statement is executed:}
\begin{verbatim}
CREATE DATABASE payroll
\end{verbatim}
Which two statements are true? (Choose two.)
\begin{choices}
	\CorrectChoice The PAYROLL database will have a page size of 4KB.
	\choice The PAYROLL database will have a page size of 8KB.
	\CorrectChoice The PAYROLL database will be an automatic storage database.
	\choice The PAYROLL database will not be an automatic storage database.
	\choice The PAYROLL database will be assigned the comment "PAYROLL DATABASE."
\end{choices}

\begin{solution}
When the simplest form of the CREATE DATABASE command is executed, the database will be
\begin{itemize}
	\item an automatic storage database
	\item have a page size of 4 KB
	\item be created on the default database path that is specified in the \textit{dftdbpath} database manager configuration parameter
	\item default table spaces (SYSCATSPACE, TEMPSPACE1, and USERSPACE1) will be automatic storage table spaces (Their containers will also be created on
	the default database path.)
\end{itemize}	
---------------------

\begin{itemize}
	\item \textit{Answer B} PAGESIZE 8192 option
	\item \textit{Answer C} AUTOMATIC STORAGE NO option
	\item \textit{Answer D} WITH "PAYROLL DATABASE" option 
\end{itemize}
\end{solution}

\newpage
%\fullwidth{\Large \textbf{Working with DB2 Data Using SQL}}
\section{Working with DB2 Data Using SQL}
\question[1]
If the following result set is desired:
\begin{verbatim}
STATE          REGION          AVG_INCOME
------------- --------------- -------------
MARYLAND      MID-ATLANTIC     86056.00
NEW JERSEY    NORTHEAST        85005.00
CONNECTICUT   NORTHEAST        84558.00
MASSACHUSETTS NORTHEAST        82009.00
ALASKA        PACIFIC-ALASKA   79617.00
\end{verbatim}
Which SQL statement must be executed?
\begin{choices}
\choice \begin{verbatim}
		SELECT state, region, avg_income
		  FROM census_data
		  ORDER BY 3
		  FETCH FIRST 5 ROWS
		\end{verbatim}
\choice \begin{verbatim}
		SELECT state, region, avg_income
		  FROM census_data
		  ORDER BY 3
		  FETCH FIRST 5 ROWS ONLY
	    \end{verbatim}
\choice \begin{verbatim}
		SELECT state, region, avg_income
		  FROM census_data
		  ORDER BY 3 DESC
		  FETCH FIRST 5 ROWS
		\end{verbatim}
\CorrectChoice \begin{verbatim}
		SELECT state, region, avg_income
		  FROM census_data
		  ORDER BY 3 DESC
		  FETCH FIRST 5 ROWS ONLY
        \end{verbatim}
\end{choices}

\question[1]
Which type of join will usually produce the smallest result set?
\begin{choices}
\CorrectChoice INNER JOIN
\choice LEFT OUTER JOIN
\choice RIGHT OUTER JOIN
\choice FULL OUTER JOIN
\end{choices}

\newpage
\question[1]
{\color{red}Which statement about SQL subqueries is NOT true?}
\begin{choices}
\choice A subquery can be used with an UPDATE statement to supply values for one or more columns 
that are to be updated.
\choice A subquery can be used with an INSERT statement to retrieve values from one base table or view
and copy them to another.
\CorrectChoice If a subquery is used with a DELETE statement and the result set produced is empty, every record
will be deleted from the table specified.
\choice If a subquery is used with an UPDATE statement and the result set produced contains multiple rows,
the operation will fail and an error will be generated.
\end{choices}

\begin{solution}
\begin{itemize}
\item \textit{Answer C} If a subquery is used with a DELETE statement and the result set produced is
empty, \textbf{no} records will be deleted from the table specified and \textbf{an error will be
returned}.
\item \textit{Answer D} when used with an UPDATE statement, the subselect provided must \textbf{not
 return more than one row}.
\end{itemize}
\end{solution}


\question[1]
{\color{red} Which statement about savepoints is NOT true?}
\begin{choices}
\choice You can use as many savepoints as you desire within a single unit of work, provided you do not
nest them.
\choice Savepoints provide a way to break the work being done by a single large transaction into one or
more smaller subsets.
\CorrectChoice The COMMIT FROM SAVEPOINT statement is used to commit a subset of database changes that have
been made within a unit of work.
\choice The ROLLBACK TO SAVEPOINT statement is used to back out a subset of database changes that have
been made within a unit of work.
\end{choices}

\begin{solution}
\begin{itemize}
\item \textit{Answer C} TO SAVEPOINT clause cannot be used with a COMMIT statement to apply a
subset of database changes that have been made by a transaction to a database and make them permanent.
\end{itemize}
\end{solution}

\newpage
\question[1]
Which two statements about UPDATE processing are true? (Choose two.)
\begin{choices}
\choice A positioned UPDATE is used to modify one or more rows, and a searched UPDATE is used to 
modify exactly one row.
\CorrectChoice A searched UPDATE is used to modify one or more rows, and a positioned UPDATE is used to
modify exactly one row.
\choice When the UPDATE statement modifies parent key columns, the values of corresponding foreign key
columns are modified as well.
\choice The UPDATE statement can be used to remove data from specified columns in the rows of a table,
provided those columns are not nullable.
\CorrectChoice The UPDATE statement can be used to modify the values of specified columns in the rows
of a table, view, or underlying table(s) of a specified fullselect.
\end{choices} 

\begin{solution}
\begin{itemize}
\item \textit{Answer D} are \sout{not} nullable
\item \textit{Answer C} When the UPDATE statement is used to modify the values stored in parent key 
columns, the update rule of the corresponding referential constraint is evaluated to determine whether
the update is allowed - the values of corresponding foreign key columns are \textit{not altered}.
\end{itemize}
\end{solution}

\question[1]
A table named SALES has two columns: SALES\_AMT and REGION\_CD. Which SQL statement will return the number
of sales in each region, ordered by number of sales made?
\begin{choices}
\choice \begin{verbatim}
		SELECT sales_amt, COUNT(*)
		  FROM sales
		  ORDER BY 2
	 	\end{verbatim}
\choice \begin{verbatim}
		SELECT sales_amt, COUNT(*)
		  FROM sales
		  GROUP BY sales_amt
		  ORDER BY 1
		\end{verbatim}
\CorrectChoice \begin{verbatim}
		SELECT region_cd, COUNT(*)
		  FROM sales
		  GROUP BY region_cd
		  ORDER BY COUNT(*)
		\end{verbatim}
\choice \begin{verbatim}
		SELECT region_cd, COUNT(*)
		  FROM sales
		  GROUP BY sales_amt
		  ORDER BY COUNT(*)
		\end{verbatim}
\end{choices}

\newpage
\question[1]
A user wants to retrieve records from a table named SALES that satisfy at least one of the following
criteria:
	\begin{itemize}
	\item The sales date (SALESDATE) is after June 1, 2012, and the sales amount (AMT) is greater
	than \$40.00.
	\item The sales was made in the hardware department. 
	\end{itemize}
Which SQL statement will accomplish this?
\begin{choices}
\choice \begin{verbatim}
		SELECT * FROM sales
		  WHERE (salesdate > '2012-06-01' OR (amt > 40
		   AND (dept = 'Hardware')
		\end{verbatim}
\choice \begin{verbatim}
		SELECT * FROM sales
		  WHERE (salesdate > '2012-06-01') OR (amt > 40)
		    OR (dept = 'Hardware')
		\end{verbatim}
\choice \begin{verbatim}
 		SELECT * FROM sales
 		  WHERE (salesdate > '2012-06-01' AND amt > 40
 		   AND (dept = 'Hardware')
		\end{verbatim}
\CorrectChoice \begin{verbatim}
		SELECT * FROM sales
		  WHERE (salesdate > '2012-06-01' AND amt > 40)
		   OR (dept = 'Hardware')
		\end{verbatim}
\end{choices}

\question[1]
{\color{red} Which two statements about INSERT operations are true?} (Choose two.)
\begin{choices}
\choice The INSERT statement can be used to insert rows into a table, view, or table function.
\CorrectChoice Inserted values must satisfy the conditions of any check constraints that have been defined 
on the table specified.
\CorrectChoice The values provided in the VALUES clause of an INSERT statement are assigned to columns in 
the order in which they appear.
\choice If an INSERT statement omits any column from the inserted row that is defined as NULL or NOT
NULL WITH DEFAULT, the statement will fail.
\choice If the underlying table of a view that is referenced by an INSERT statement has one or more
unique indexes, each row inserted does not have to conform to the constraints imposed by those indexes.
\end{choices}

\begin{solution}
\begin{itemize}
\item \textit{Answer A} INSERT statement can be used to insert rows into a table or updatable view, 
\textit{but not a table function}
\item \textit{Answer D} omitted column must either accept NULL values (ie. have not been defined with 
a NOT NULL constraint) or have been defined with a default constraint (ie, NOT NULL WITH DEFAULT). \sout{not fail}
\item \textit{Answer E} must conform
\end{itemize}
\end{solution}

\newpage
\question[1]
Which SQL statement should be used to select the minimum and maximum rainfall amounts (RAINFALL), by 
month (MONTH), from a table named WEATHER?
\begin{choices}
\choice \begin{verbatim}
		SELECT month, MIN(rainfall), MAX(rainfall)
		  FROM weather
		  ORDER BY month
		\end{verbatim}
\CorrectChoice \begin{verbatim}
		SELECT month, MIN(rainfall), MAX(rainfall)
		  FROM weather
		  GROUP BY month
		\end{verbatim}
\choice \begin{verbatim}
		SELECT month, MIN(rainfall), MAX(rainfall)
		  FROM weather
		  GROUP BY month, MIN(rainfall), MAX(rainfall)
		\end{verbatim}
\choice \begin{verbatim}
		SELECT month, MIN(rainfall), MAX(rainfall)
		  FROM weather
		  ORDER BY month, MIN(rainfall), MAX(rainfall)
		\end{verbatim}
\end{choices}

\question[1]
An SQL function named REGIONAL\_SALES was created as follows:
	\begin{verbatim}
	CREATE FUNCTION regional_sales()
		RETURNS TABLE (region_id VARCHAR(20),
					   sales_amt DECIMAL(8,2))
		READS SQL DATA
		BEGIN ATOMIC
		RETURN
			SELECT region, amt
			FROM sales
			ORDER BY region;
		END
	\end{verbatim}
Which two statements demonstrate the proper way to use this function in a query? (Choose two.)
\begin{choices}
\choice \texttt{SELECT * FROM regional\_sales()}
\choice \texttt{SELECT regional\_sales (region\_id, sales\_amt)}
\choice \texttt{SELECT region\_id, sales\_amt FROM regional\_sales()}
\CorrectChoice \texttt{SELECT * FROM TABLE (regional\_sales()) AS results}
\CorrectChoice \texttt{SELECT region\_id, sales\_amt FROM TABLE(region\_sales()) AS results}
\end{choices}

\newpage
\question[1]
If the following SQL statement is executed:
\begin{verbatim}
SELECT dept, AVG(salary)
 FROM employee
 GROUP BY dept
 ORDER BY 2
\end{verbatim}
What will be the results?
\begin{choices}
\choice The department number and average salary for all employees will be retrieved from a table named
EMPLOYEE, and the results will be arranged in descending order, by department.
\choice The department number and average salary for all departments will be retrieved from a table named
EMPLOYEE, and the results will be arranged in ascending order, by department.
\CorrectChoice The department number and average salary for all employees will be retrieved from a table named
EMPLOYEE, and the results will be arranged in ascending order, by average departmental salary.
\choice The department number and average salary for all departments will be retrieved from a table named
EMPLOYEE, and the results will be arranged in descending order, by average departmental salary.
\end{choices}

\begin{solution}
the average salary is being calculated using data stored in the EMPLOYEE table-\textit{not in the
DEPARTMENT table}, which is where two answers imply that salary information is to be retrieved from (
\textit{Answer B} and \textit{Answer D}
\end{solution}

\question[1]
If a table named SALES contains information about invoices that do not have a negative balance, which 
two SQL statements can be used to retrieve invoice numbers for invoices that are for less than \$25,000.00
? (Choose two.)
\begin{choices}
\CorrectChoice \texttt{SELECT invoice\_num FROM sales WHERE amt < 25000}
\choice \texttt{SELECT invoice\_num FROM sales WHERE amt < 25,000}
\choice \texttt{SELECT invoice\_num FROM sales WHERE amt LESS THAN 25,000}
\CorrectChoice \texttt{SELECT invoice\_num FROM sales WHERE amt BETWEEN 0 AND 25000}
\choice \texttt{SELECT invoice\_num FROM sales WHERE amt BETWEEN 0 AND 25,000}
\end{choices}

\newpage
\question[1]
{\color{red}An SQL function designed to convert miles to kilometers was created as follows:}
\begin{verbatim}
CREATE FUNCTION mi_to_km (IN miles FLOAT)
  RETURN FLOAT
  LANGUAGE SQL
  SPECIFIC convert_mtok
  READS SQL DATA
  RETURN FLOAT (miles * 1.60934)
\end{verbatim}
How can this function be used to convert miles (MILES) values stored in a table named DISTANCES?
\begin{choices}
\choice CALL mi\_to\_km (distances.miles)
\choice CALL convert\_mtok (distances.miles)
\CorrectChoice SELECT mi\_to\_km (miles) FROM distances
\choice SELECT convert\_mtok (miles) FROM distances
\end{choices}

\begin{solution}
\begin{itemize}
\item CALL statement is used to invoke a stored procedure - not a UDF
\item Specific name that is assigned to a UDF can be used to reference or delete(drop) the UDF, but
not to invoke it
\end{itemize}
\end{solution}

\question[1]
Which two statements about system-period temporal tables are true? (Choose two.)
\begin{choices}
\choice They store user-based period information.
\choice They do not require a separate history table.
\CorrectChoice They store system-based historical information.
\CorrectChoice They can be queried without a time-period specification.
\choice They manage data based on time criteria specified by users or applications.
\end{choices} 

\question[1]
A table named PARTS contains a record of every part that has been manufactured by a company. A user
wishes to see the total number of parts that have been made by each craftsman employed at the company.
Which SQL statement will produce the desired results?
\begin{choices}
\choice \texttt{SELECT name, COUNT(*) AS parts\_made FROM parts}
\CorrectChoice \texttt{SELECT name, COUNT(*) AS parts\_made FROM parts GROUP BY name}
\choice \texttt{SELECT name, COUNT(DISTINCT name) AS parts\_made FROM parts}
\choice \texttt{SELEC DISTINCT name, COUNT(*) AS parts\_made FROM parts GROUP BY parts\_made}
\end{choices}

\newpage
\question[1]
Given an EMPLOYEES table and a DEPARTMENT table, a user wants to produce a list of all departments and
employees who work in them, including departments that no employees have been assigned to. Which SQL
statement will produce the desired list?
\begin{choices}
\choice \begin{verbatim}
		SELECT employees.name, departments.deptname
		FROM employees
		INNER JOIN department ON
		employees.dept = departments.deptno
		\end{verbatim} 
\choice	\begin{verbatim}
		SELECT employees.name, departments.deptname
		FROM employees
		INNER JOIN department ON
		departments.deptno = employees.dept
		\end{verbatim}
\choice \begin{verbatim}
		SELECT employees.name, departments.deptname
		FROM employees
		LEFT OUTER JOIN departments ON
		employees.dept = departments.deptno
		\end{verbatim}
\CorrectChoice \begin{verbatim}
		SELECT employees.name, departments.deptname,
		FROM employees
		RIGHT OUTER JOIN departments ON
		employees.dept = departments.deptno
		\end{verbatim}
\end{choices}

\question[1]
{\color{red} Which statements are NOT allowed in the body of an SQL scalar user-defined function?}
\begin{choices}
\choice CALL statements
\CorrectChoice COMMIT statements
\choice SQL CASE statements
\choice SQL control statements
\end{choices}

\begin{solution}
COMMIT and ROLLBACK statements are not allowed in the body of an SQL user-defined function (UDF).
\end{solution}

\newpage
\question[1]
A table named TABLE\_A contains 200 rows and a user wants to update the 10 rows in this table with the
lowest values in a column named COL1. Which SQL statement will produce the desired results?
\begin{choices}
\choice \begin{verbatim}
		UPDATE
		  (SELECT * FROM table_a
		    ORDER BY col1 ASC) AS temp
		  SET col2 = 99
		  FETCH FIRST 10 ROWS ONLY
		\end{verbatim}
\choice \begin{verbatim}
		UPDATE
		  (SELECT * FROM table_a
		    ORDER BY col1 DESC) AS temp
		  SET col2=99
		  FETCH FIRST 10 ROWS ONLY
		\end{verbatim}
\CorrectChoice \begin{verbatim}
		UPDATE
		  (SELECT * FROM table_a
		    ORDER BY col1 ASC
	  		FETCH FIRST 10 ROWS ONLY) AS temp
	  	  SET col2=99
		\end{verbatim}
\choice \begin{verbatim}
		UPDATE
		  (SELECT * FROM table_a
		    ORDER BY col1 DESC
		    FETCH FIRST 10 ROWS ONLY) AS temp
		  SET col2 = 99
		\end{verbatim}
\end{choices}

\question[1]
{\color{red} Which statement best descirbes a transaction?}
\begin{choices}
\choice A transaction is a recoverable sequence of operations whose point of consistency is established
only when a savepoint is created.
\choice A transaction is recoverable sequence of operations whose point of consistency can be obtained
by querying the system catalog tables.
\choice A transaction is a recoverable sequence of operations whose point of consistency is established
when a database connection is established or a savepoint is created.
\CorrectChoice A transaction is recoverable sequence of operations whose point of consistency is established
when an executable SQL statement is processed after a database connection has been established or a
previous transaction has been terminated.
\end{choices}

\begin{solution}
\begin{itemize}
\item \textit{Answer A} and \textit{Answer C} savepoints do not define points of consistency for a transaction
\item \textit{Answer B} System catalog has nothing to do with defining points of consistency for
transactions
\end{itemize}
\end{solution}

\newpage
\question[1]
A table named TABLE\_A contains 200 rows and a user wants to delete
the last 10 rows from this table. Which SQL statement will produce the
desired results?
\begin{choices}
\choice \begin{verbatim}
		DELETE FROM
		 (SELECT * FROM table_a 
		   ORDER BY col1 ASC
		   FETCH FIRST 10 ROWS ONLY) AS result
		\end{verbatim}
\CorrectChoice \begin{verbatim}
		DELETE FROM
		  (SELECT * FROM table_a
			ORDER BY col1 DESC
			FETCH FIRST 10 ROWS ONLY) AS result
		\end{verbatim}
\choice \begin{verbatim}
		DELETE FROM
		 (SELECT * FROM table_a
		   ORDER BY col1 ASC
		   FETCH LAST 10 ROWS ONLY) AS result
		\end{verbatim}
\choice \begin{verbatim}
		DELETE FROM
		 (SELECT * FROM table_a
		   ORDER BY col1 DESC
		   FETCH LAST 10 ROWS ONLY) AS result
		\end{verbatim}
\end{choices}

\begin{solution}
no FETCH LAST ... ROWS ONLY clause
\end{solution}
\question[1]
Which clause could be added to the following SQL statement
\begin{verbatim}
SELECT student_id, enroll_date, gpa
 FROM students
\end{verbatim}
to ensure that only information (STUDENT\_ID, ENROLL\_DATE, and
GPA) for students who started school before 2012 and who have a GPA
that is higher than 3.50 will be retrieved?
\begin{choices}
\choice FOR enroll\_date < '2012-01-01' OR gpa>3.50
\choice FOR enroll\_date < '2012-01-01' AND gpa>3.50
\CorrectChoice WHERE enroll\_date < '2012-01-01' OR gpa>3.5
\choice	WHERE enroll\_date < '2012-01-01' AND gpa>3.5
\end{choices}

\question[1]
{\color{red}Which statement correctly describes what a native SQL stored procedure is?}
\begin{choices}
\CorrectChoice A procedure whose body is written entirely in SQL or SQL PL.
\choice A procedure that is written in a high-level programming language such as Java or REXX.
\choice A procedure whose body is written entirely in SQL, but that is implemented as an external program.
\choice	A procedure that accesses data using an Object Linking and Embedding, Database (OLE DB) provider.
\end{choices}

\begin{solution}
\begin{itemize}
\item \textit{Answer B} External stored procedure
\item \textit{Answer C} External SQL stored procedure
\item \textit{Answer D} Cannot develop a stored procedure that accesses data using an Object Linking and 
Embedding, Database (OLE DB) provider - to access data in this manner, you must develop an OLE DB External
Table function.
\end{itemize}
\end{solution}

\question[1]
{\color{red}Given the following statements}:
\begin{verbatim}
CREATE TABLE customer (custid INTEGER, custinfo XML);
INSERT INTO customer VALUES (100,
'<customerinfo>
  <name>ACME Manufacturing</name>
  <addr country="United States">
     <street>25 Elm Street</street>
 <city>Raleigh</city>
 <state>North Carolina</state>
 <zip>27603</zip>
  </addr>
</customerinfo>');
\end{verbatim}
If the following XQuery statement is executed:
\begin{verbatim}
XQUERY
for $info in db2-fn:xmlcolumn('CUSTOMER.CUSTINFO')/customerinfo
return $info/name
\end{verbatim}
What will be the result?
\begin{choices}
\choice \texttt{ACME Manufacturing}
\CorrectChoice \texttt{<name>ACME Manufacturing</name>}
\choice \texttt{<customerinfo>ACME Manufacturing</customerinfo>}
\choice \texttt{<customerinfo><name>ACME Manufacturing</name></customerinfo>}
\end{choices}

\begin{solution}
The FLOWER XQuery expression is comparable to the SELECT-FROM-WHERE statement/clause combination available
with SQL. The basic syntax for a FLOWER expression is:
\begin{verbatim}
XQUERY
	for $Variable1 IN Expression1
	let $Variable2 := Expression2
	where Expression3
	order by Expression4 [ASCENDING | DESCENDING]
	return Expression3
\end{verbatim}
Consequently, the XQuery statement used in the scenario presented will return the expression 
"<name>ACME Manufacturing</name>". This expression is obtained by searching the XML data value
stored in the CUSTINFO column of a table named CUSTOMER for the first opening tag found under the 
customerinfo root (outermost) element that is followed by a value. 

Had the text() function been used with the XQuery statement, the opening and closing tags for the name 
would have been removed and the value "ACME Manufacturing" would have been returned (\textit{Answer A})
Because the customerinfo element is the root element of the XML data value presented, it is typically
referenced in the for expression of the XQuery statement and is not returned as part of the return
expression(\textit{Answer C} and \textit{Answer D})
\end{solution}

\question[1]
{\color{red}Which two statements about roll back operations are correct? }(Choose two.)
\begin{choices}
\CorrectChoice When a ROLLBACK statement is executed, all locks held by the terminating transaction are released.
\choice When a ROLLBACK TO SAVEPOINT statement is executed, all locks acquired after the most recent
savepoint are released.
\choice When a ROLLBACK statement is executed, all locks acquired for open cursors that were 
declared WITH HOLD are held.
\choice When a ROLLBACK TO SAVEPOINT statement is executed, all locks acquired up to the most recent
savepoint are released.
\CorrectChoice When a ROLLBACK TO SAVEPOINT statement is executed, a savepoint is not automatically deleted as
part of the rollback operation.
\end{choices}

\begin{solution}
\begin{itemize}
\item \textit{Answer C} are released
\item \textit{Answer B} When a ROLLBACK TO SAVEPOINT statement is executed, all locks acquired after the
savepoint specified was created are released-\textit{not the locks acquired after the most recent 
savepoint}
\item \textit{Answer D} \textit{nor the locks acquired up to the most recent savepoint}
\end{itemize}
\end{solution}

\newpage
\question[1]
Which SQL statement illustrates the proper way to perform a positioned update operation on a table named
SALES?
\begin{choices}
\choice \texttt{UPDATE sales SET amt = 102.45}
\choice \texttt{UPDATE sales SET amt = 102.45 WHERE cust\_id='000290'}
\CorrectChoice \texttt{UPDATE sales SET amt = 102.45 WHERE CURRENT OF cursor1}
\choice \begin{verbatim}
		UPDATE sales SET amt = 102.45
		WHERE ROWID = (SELECT ROWID FROM sales WHERE cust_id='000290')
		\end{verbatim}
\end{choices}

\question[1]
Which two statements about application-period temporal tables are true? (Choose two.)
\begin{choices}
\CorrectChoice They are useful when one wants to keep user-based time period information.
\choice They consist of explicitly supplied timestamps and a separate associated history table.
\choice They are useful when one wants to keep both user-based and system-based time period information.
\CorrectChoice They are based on explicitly supplied timestamps that define the time periods during which
data is valid.
\choice They consist of a pair of columns with database-manager maintained values that indicate the 
period when a row is current.
\end{choices}

\begin{solution}
\begin{itemize}
\item \textit{Answer E} System-period temporal table
\end{itemize}
\end{solution}

\question[1]
When should the TRUNCATE statement be used?
\begin{choices}
\CorrectChoice When you want to delete all rows from a table without generating log records.
\choice When you want to delete select rows from a table without generating log records.
\choice When you want to delete all rows from a table and fire any delete triggers that have been 
defined for the table.
\choice When you want to delete select rows from a table and fire any delete triggers that have been
defined for the table.
\end{choices}

\begin{solution}
\begin{itemize}
\item If IGNORE DELETE TRIGGERS clause is specified with the TRUNCATE statement used, DELETE tiggers that 
have been defined on the table \textit{will not be fired} as the data in the table is deleted
\item If RESTRICT WHEN DELETE TRIGGERS clause is used instead, DB2 will examine the system catalog to 
determine whether DELETE triggers on the table exist; if one or more triggers are found, the truncate
operation will fail and error will be returned.
\end{itemize}

\end{solution}
\question[1]
Which two commands will terminate the current transaction and start a new transaction boundary?
(Choose two.)
\begin{choices}
\CorrectChoice COMMIT
\choice REFRESH
\choice RESTART
\choice CONNECT
\CorrectChoice ROLLBACK
\end{choices}

\question[1]
Which SQL statement should be used to retrieve the minimum and maximum annual temperature (TEMP) for
each major city (CITY), sorted by city, from a table named WEATHER?
\begin{choices}
\choice \begin{verbatim}
		SELECT city, MIN(temp), MAX(temp)
		  FROM weather
		  ORDER BY city
		\end{verbatim}
\choice \begin{verbatim}
		SELECT city, MIN(temp), MAX(temp)
		  FROM weather
		  GROUP BY city
		\end{verbatim}
\CorrectChoice \begin{verbatim}
		SELECT city, MIN(temp), MAX(temp)
		  FROM weather
		  GROUP BY city
		  ORDER BY city
		\end{verbatim}
\choice \begin{verbatim}
		SELECT city, MIN(temp), MAX(temp)
		  FROM weather
		  GROUP BY MIN(temp), MAX(temp)
		  ORDER BY city
		\end{verbatim}
\end{choices}

\question[1]
{\color{red}What is the XMLTABLE() function typically used for?}
\begin{choices}
\choice To convert a well-formed XML document into a table of character string values.
\CorrectChoice To obtain values from XML documents that are to be inserted into one or more tables.
\choice To parse a character string value and return a table of well-formed XML documents.
\choice To produce a temporary table whose columns are based on the elements found in a well-formed XML
document.
\end{choices}

\begin{solution}
\begin{itemize}
\item \textit{Answer A} XMLSERIALIZE()
\item \textit{Answer C} XMLPARSE()
\item \textit{Answer D} No such XML function exist; need to construct a query or XQuery expression to
perform this type of operation
\end{itemize}
\end{solution}
% Question with parts
% \newpage
% \addpoints
% \question Consider the function $f(x)=x^2$.
% \begin{parts}
% \part[5] Find $f'(x)$ using the limit definition of derivative.
% \vfill
% \part[5] Find the line tangent to the graph of $y=f(x)$ at the point where $x=2$.
% \vfill
% \end{parts}

% If you want the total number of points for a question displayed at the top,
% as well as the number of points for each part, then you must turn off the point-counter
% or they will be double counted.
% \newpage
% \addpoints
% \question[10] Consider the function $f(x)=x^3$.
% \noaddpoints % If you remove this line, the grading table will show 20 points for this problem.
% \begin{parts}
% \part[5] Find $f'(x)$ using the limit definition of derivative.
% \vspace{4.5in}
% \part[5] Find the line tangent to the graph of $y=f(x)$ at the point where $x=2$.
% \end{parts}



\end{questions}
\end{document}
