% Exam Template for UMTYMP and Math Department courses
%
% Using Philip Hirschhorn's exam.cls: http://www-math.mit.edu/~psh/#ExamCls
%
% run pdflatex on a finished exam at least three times to do the grading table on front page.
%
%%%%%%%%%%%%%%%%%%%%%%%%%%%%%%%%%%%%%%%%%%%%%%%%%%%%%%%%%%%%%%%%%%%%%%%%%%%%%%%%%%%%%%%%%%%%%%

% These lines can probably stay unchanged, although you can remove the last
% two packages if you're not making pictures with tikz.
\documentclass[answers, 11pt]{exam}
\RequirePackage{amssymb, amsfonts, amsmath, latexsym, verbatim, xspace, setspace}
\RequirePackage{tikz, pgflibraryplotmarks}

% By default LaTeX uses large margins.  This doesn't work well on exams; problems
% end up in the "middle" of the page, reducing the amount of space for students
% to work on them.
\usepackage[margin=1in]{geometry}
\usepackage{color}

% Customize the Solution box
\renewcommand{\solutiontitle}{\noindent\textbf{Explanation:}\par\noindent}

% Here's where you edit the Class, Exam, Date, etc.
\newcommand{\class}{IBM Exam 610 Question Bank}
% \newcommand{\term}{Spring 2012}
% \newcommand{\examnum}{Exam 2}
\newcommand{\examdate}{2/8/15}
\newcommand{\timelimit}{90 Minutes}

% For an exam, single spacing is most appropriate
\singlespacing
% \onehalfspacing
% \doublespacing

% For an exam, we generally want to turn off paragraph indentation
\parindent 0ex

\begin{document} 

% These commands set up the running header on the top of the exam pages
\pagestyle{head}
\firstpageheader{}{}{}
\runningheader{\class}{Page \thepage\ of \numpages}{\examdate}
\runningheadrule

\begin{flushright}
\begin{tabular}{p{2.8in} r l}
\textbf{\class} & \\
% \textbf{\term} &&\\
% \textbf{\examnum} &&\\
\textbf{\examdate} &  \\
\textbf{Time Limit: \timelimit} & \textbf{Name (Print):} & \makebox[2in]{\hrulefill}
\end{tabular}\\
\end{flushright}
\rule[1ex]{\textwidth}{.1pt}


This question bank contains \numpages\ pages (including this cover page) and
\numquestions\ questions. 

% You may \textit{not} use your books, notes, or any calculator on this exam.\\

% You are required to show your work on each problem on this exam.  The following rules apply:\\

\begin{minipage}[t]{3.7in}
\vspace{0pt}
\begin{itemize}

\item IBM Exam 610 corresponds to \textbf{Test C2090-610: DB2 10.1 Fundamentals}

\item The questions adapted from the following sources:

\begin{itemize}
\item \textbf{DB 2 10.1 Fundamentals Certification Study Guide} by Roger E. Sanders
\end{itemize}
% \item \textbf{If you use a ``fundamental theorem'' you must indicate this} and explain
% why the theorem may be applied.

% \item \textbf{Organize your work}, in a reasonably neat and coherent way, in
% the space provided. Work scattered all over the page without a clear or

% \item \textbf{Mysterious or unsupported answers will not receive full
% credit}.  A correct answer, unsupported by calculations, explanation,
% or algebraic work will receive no credit; an incorrect answer supported
% by substantially correct calculations and explanations might still receive
% partial credit.


% \item If you need more space, use the back of the pages; clearly indicate when you have done this.
\end{itemize}

% Do not write in the table to the right.
\end{minipage}
\hfill
\begin{minipage}[t]{2.3in}
\vspace{0pt}
%\cellwidth{3em}
\gradetablestretch{2}
\vqword{Question}
\addpoints % required here by exam.cls, even though questions haven't started yet.	
\gradetable[v][pages]  % Use [pages] to have grading table by page instead of question

\end{minipage}
\newpage % End of cover page

%%%%%%%%%%%%%%%%%%%%%%%%%%%%%%%%%%%%%%%%%%%%%%%%%%%%%%%%%%%%%%%%%%%%%%%%%%%%%%%%%%%%%
%
% See http://www-math.mit.edu/~psh/#ExamCls for full documentation, but the questions
% below give an idea of how to write questions [with parts] and have the points
% tracked automatically on the cover page.
%
%
%%%%%%%%%%%%%%%%%%%%%%%%%%%%%%%%%%%%%%%%%%%%%%%%%%%%%%%%%%%%%%%%%%%%%%%%%%%%%%%%%%%%%

\begin{questions}

\fullwidth{\Large \textbf{Planning}}

% Basic question
\addpoints
% \question[10] Differentiate $f(x)=x^2$ with respect to $x$.
\question[1]
A database will be used primarily to identify sales patterns
for products sold within the last three years and to summarize
sales by region, on a quarterly basis. Which type of system is 
needed?
\begin{choices}
\choice Analytical
\choice DB2 pureScale
\CorrectChoice Data warehouse
\choice Online transaction processing (OLTP)
\end{choices}

\begin{solution}
Data warehouses (\textit{Correct Answer C}) are typically used to store and manage large volumes of data that is often
historical in nature and that is used primarily for analysis. Thus, a data warehouse could be used to identify sales patterns
for products sold within the past three years or to summarize sales by region, on a quarterly basis. \textit{InfoSphere Warehouse} is offered for data warehouse.
\par

Online transaction processing (OLTP) systems (\textit{Answer D}), are designed to support day-to-day, mission-critical business
activities such as web-based order entry and stock trading. \textit{DB2 pureScale feature} is offered for OLTP workloads. 
\par

{\color{red} Analytical} workloads (\textit{Answer A}) are better handled by a specialized product
known as \textit{DB2 for i and by IBM BLU Acceleration}, which is currently available
only with DB2 10.5 for LUW.

\end{solution}

\question[1]
Which product can be used to tune performance for a single query?
\begin{choices}
\CorrectChoice IBM Data Studio
\choice IBM Control Center
\choice IBM Data Administrator
\choice IBM Workload Manager
\end{choices}

\begin{solution}
\textit{IBM Data Studio} (\textit{Correct Answer A}) is an Eclipse-based, integrated
development environment (IDE) that can be used to perform instance and database
administration, routine (SQL procedure, SQL functions, etc.) and application
development, and performance-tuning tasks. It replaces the \textit{DB2 Control Center} 
(\textit{Answer B}) as the standard GUI tool for DB2 database administration and
application development.
\par

\textit{IBM Workload Manager}, or WLM (\textit{Answer D}) is a comprehensive workload
management feature that can help identify, manage, and control database workloads to
maximize database server \textcolor{red}{throughput} and \textcolor{red}{resource
 utilization}.
\par

There is \textbf{NO} such product as IBM Data Administrator (\textit{Answer C}).

\end{solution}

\newpage
\addpoints

\question[1]
Which two DB2 products are suitable for very large data warehouse applications? (Choose two.)
\begin{choices}
\choice DB2 for i
\CorrectChoice DB2 for AIX
\CorrectChoice DB2 for z/OS
\choice DB2 pureScale
\choice DB2 Express-C
\end{choices}

\begin{solution}
\textit{DB2 for i} (\textit{Answer A}), formerly known as DB2 for i5/OS, is an
advanced, 64-bit Relational Database Management System that leverages the high
performance, virtualization, and energy efficiency features of IBM's Power Systems; 
its self-managing attributes, security, and built-in analytical processing functions
make \textit{DB2 for i} an ideal database server for {\color{red} applications that 
are analytical in nature}.
\par

\textit{DB2 for z/OS} (\textit{Correct Answer C} is a multiuser, full-function
database management system that has been designed specifically for z/OS, 
IBM's flagship mainframe operating system. Tightly integrated with the IBM mainframe,
\textit{DB2 for z/OS} leverages the strengths of System z 64-bit architecture to
provide, among other things, the ability to support complex {\color{red} data
warehouses}. 
\par

In addition to DB2 for z/OS, all of the DB2 Editions available \textbf{except} 
\textit{DB2 Express-C} (\textit{Answer E}) and \textit{DB2 Express Edition} can be
used to create data warehouse and OLTP environments. \textbf{However},
IBM offers two solutions that are tailored specifically for one workload type or
the other: \textit{InfoSphere Warehouse} for {\color{red} data warehousing} workloads
and the \textit{DB2 pureScale Feature} (\textit{Answer D}) for {\color{red} OLTP}
workloads.

\end{solution}


\question[1]
What is the DB2 Workload Manager (WLM) used for?
\begin{choices}
\choice To identify, diagnose, solve, and prevent performance problems in DB2 products and associated applications.
\CorrectChoice To customize execution environments for the purpose of controlling system resources so that one
		department or service class does not overwhelm the system.
\choice To respond to significant changes in a database's workload by dynamically distributing available memory resources
		among several different database memory consumers.
\choice To improve the performance of applications that require frequent, but relatively transient, simultaneous user
		connections by allocating host database resources only for the duration of an SQL transaction.
\end{choices}

\begin{solution}
\textit{IBM InfoSphere Optim Performance Manager Extended Edition} can be used to 
identify, diagnose, solve, and prevent performance problems in DB2 products and
associated applications (\textit{Answer A})
\par

With \textit{DB2 Workload Manager}, it is possible to customize execution
environments so that no single workload can control and consume all of the system
resources available. (This prevents any one department or service class from
overwhelming the system.) (\textit{Correct Answer B})
\par

The \textit{Self-Tuning Memory Manager (STMM)} responds to significant changes in
a database's workload by dynamically distributing available memory resources among
several different database memory consumers. (\textit{Answer C})
\par

The \textit{Connection Concentrator} improves the performance of applications that
require frequent, but relatively transient, simultaneous user connections by
allocating host database resources only for the duration of an SQL transaction. 
(\textit{Answer D})

\end{solution}


\question[1]
Which of the following is NOT a characteristic of a data warehouse?
\begin{choices}
\choice Sub-second response time
\choice Voluminous historical data
\choice Heterogeneous data sources
\choice Summarized queries that perform aggregations and joins
\end{choices}

\newpage
\addpoints
\question[1]
Which statement about the DB2 pureScale feature is NOT true?
\begin{choices}
\choice The DB2 pureScale feature provides a database cluster solution for nonmainframe platforms.
\choice The DB2 pureScale feature is only available as part of DB2 Advanced Enterprise Server Edition.
\choice The DB2 pureScale feature can only work with the General Parallel File System (GPFS) file system.
\choice The DB2 pureScale feature is best suited for online transaction processing (OLTP) workloads.
\end{choices}

\question[1]
Which two statements about large object (LOB) locators are true? (Choose two.)
\begin{choices}
\choice A LOB locator represents a value for a LOB resource that is stored in a database.
\choice A LOB locator is a simple token value that is used to refer to a much bigger LOB value.
\choice A LOB locator is a special data type that is used to store LOB data in external binary files.
\choice A LOB locator represents a value for a LOB resource that is stored in an external binary file.
\choice A LOB locator is a mechanism that acts similar to an index in the way that is organizes LOB values
		so they can be quickly located in response to a query.
\end{choices}

\question[1]
Which type of database workload typically involves making changes to a small number of records within
a single transaction?
\begin{choices}
\choice Decision support
\choice Data warehousing
\choice Online analytical processing (OLAP)
\choice Online transaction processing (OLTP)
\end{choices}

\question[1]
Which of the following is NOT a characteristic of an OLTP database?
\begin{choices}
\choice Current data
\choice Frequent updates
\choice Granular transactions
\choice Optimized for queries
\end{choices}

\question[1]
Which two platforms support DB2 10.1 pureScale environments? (Choose two.)
\begin{choices}
\choice IBM mainframes running z/OS
\choice IBM p Series servers running AIX
\choice IBM p Series servers running Linux
\choice IBM x Series servers running Linux
\choice IBM x Series servers running a supported version of Windows
\end{choices}

\newpage
\addpoints
\question[1]
Which tool can analyze and provide recommendations for tuning individual queries?
\begin{choices}
\choice IBM InfoSphere Data Architect
\choice IBM InfoSphere Optim Query Tuner
\choice IBM InfoSphere Optim pureQuery Runtime
\choice IBM InfoSphere Optim Performance Manager Extended Edition
\end{choices}

\question[1]
Which SQL statement will create a table named EMPLOYEE that can be used to store XML data?
\begin{choices}
\choice CREATE TABLE employee (empid INT, resume XML)
\choice CREATE TABLE employee (empid INT, resume XML(2000))
\choice CREATE TABLE employee (empid INT, resume CLOB AS XML)
\choice CREATE TABLE employee (empid INT, resume CLOB USING XML)
\end{choices}

\question[1]
What DB2 product provides a complete data warehousing solution that contains components
that facilitate data warehouse construction and administration?
\begin{choices}
\choice DB2 pureScale Feature
\choice DB2 Workload Manager
\choice IBM InfoSphere Warehouse
\choice Database Partitioning Feature
\end{choices}

\question[1]
Which statement about IBM Data Studio is NOT true?
\begin{choices}
\choice The IBM Data Studio administration client can be installed on servers running Red Hat Linux, SUSE Linux,
		Windows, and AIX.
\choice IBM Data Studio replaces the DB2 Control Center as the standard GUI interface for DB2 database administration
		and application development.
\choice IBM Data Studio is an Eclipsed-based, integrated development environment (IDE) that can be used to perform
		instance and database administration.
\choice IBM Data Studio allows users to connect to a DB2 database using a wizard; however, users are required to provide
		login credentials before a connection will be established.
\end{choices}

\question[1]
Which statement about inline large objects (LOBs) is NOT true?
\begin{choices} 
\choice When a table contains columns with inline LOBs, fewer rows can fit on a page.
\choice Inline LOBs are created by appending the INLINE LENGTH clause to a LOB column's definition.
\choice Because DML operations against inline LOBs are never logged, their use can reduce logging overhead.
\choice Inline LOBs improve query performance by storing LOB data in the same data pages as the rest of a table's rows,
		rather than in a separate LOB storage object.
\end{choices}

\newpage
\addpoints
\fullwidth{\Large \textbf{Security}}
% Question with parts
% \newpage
% \addpoints
% \question Consider the function $f(x)=x^2$.
% \begin{parts}
% \part[5] Find $f'(x)$ using the limit definition of derivative.
% \vfill
% \part[5] Find the line tangent to the graph of $y=f(x)$ at the point where $x=2$.
% \vfill
% \end{parts}

% If you want the total number of points for a question displayed at the top,
% as well as the number of points for each part, then you must turn off the point-counter
% or they will be double counted.
% \newpage
% \addpoints
% \question[10] Consider the function $f(x)=x^3$.
% \noaddpoints % If you remove this line, the grading table will show 20 points for this problem.
% \begin{parts}
% \part[5] Find $f'(x)$ using the limit definition of derivative.
% \vspace{4.5in}
% \part[5] Find the line tangent to the graph of $y=f(x)$ at the point where $x=2$.
% \end{parts}



\end{questions}
\end{document}