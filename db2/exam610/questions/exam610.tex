% Exam Template for UMTYMP and Math Department courses
%
% Using Philip Hirschhorn's exam.cls: http://www-math.mit.edu/~psh/#ExamCls
%
% run pdflatex on a finished exam at least three times to do the grading table on front page.
%
%%%%%%%%%%%%%%%%%%%%%%%%%%%%%%%%%%%%%%%%%%%%%%%%%%%%%%%%%%%%%%%%%%%%%%%%%%%%%%%%%%%%%%%%%%%%%%

% These lines can probably stay unchanged, although you can remove the last
% two packages if you're not making pictures with tikz.
\documentclass[11pt]{exam}
\RequirePackage{amssymb, amsfonts, amsmath, latexsym, verbatim, xspace, setspace}
\RequirePackage{tikz, pgflibraryplotmarks}

% By default LaTeX uses large margins.  This doesn't work well on exams; problems
% end up in the "middle" of the page, reducing the amount of space for students
% to work on them.
\usepackage[margin=1in]{geometry}


% Here's where you edit the Class, Exam, Date, etc.
\newcommand{\class}{IBM Exam 610 Question Bank}
% \newcommand{\term}{Spring 2012}
% \newcommand{\examnum}{Exam 2}
\newcommand{\examdate}{2/8/15}
\newcommand{\timelimit}{90 Minutes}

% For an exam, single spacing is most appropriate
\singlespacing
% \onehalfspacing
% \doublespacing

% For an exam, we generally want to turn off paragraph indentation
\parindent 0ex

\begin{document} 

% These commands set up the running header on the top of the exam pages
\pagestyle{head}
\firstpageheader{}{}{}
\runningheader{\class}{Page \thepage\ of \numpages}{\examdate}
\runningheadrule

\begin{flushright}
\begin{tabular}{p{2.8in} r l}
\textbf{\class} & \\
% \textbf{\term} &&\\
% \textbf{\examnum} &&\\
\textbf{\examdate} &  \\
\textbf{Time Limit: \timelimit} & \textbf{Name (Print):} & \makebox[2in]{\hrulefill}
\end{tabular}\\
\end{flushright}
\rule[1ex]{\textwidth}{.1pt}


This question bank contains \numpages\ pages (including this cover page) and
\numquestions\ questions. 

% You may \textit{not} use your books, notes, or any calculator on this exam.\\

% You are required to show your work on each problem on this exam.  The following rules apply:\\

\begin{minipage}[t]{3.7in}
\vspace{0pt}
\begin{itemize}

\item IBM Exam 610 corresponds to \textbf{Test C2090-610: DB2 10.1 Fundamentals}

\item The questions adapted from the following sources:

\begin{itemize}
\item \textbf{DB 2 10.1 Fundamentals Certification Study Guide} by Roger E. Sanders
\end{itemize}
% \item \textbf{If you use a ``fundamental theorem'' you must indicate this} and explain
% why the theorem may be applied.

% \item \textbf{Organize your work}, in a reasonably neat and coherent way, in
% the space provided. Work scattered all over the page without a clear or

% \item \textbf{Mysterious or unsupported answers will not receive full
% credit}.  A correct answer, unsupported by calculations, explanation,
% or algebraic work will receive no credit; an incorrect answer supported
% by substantially correct calculations and explanations might still receive
% partial credit.


% \item If you need more space, use the back of the pages; clearly indicate when you have done this.
\end{itemize}

% Do not write in the table to the right.
\end{minipage}
%\hfill
%\begin{minipage}[t]{2.3in}
%\vspace{0pt}
%\cellwidth{3em}
%\gradetablestretch{2}
%\vqword{Question}
%\addpoints % required here by exam.cls, even though questions haven't started yet.	
%\gradetable[v]  % Use [pages] to have grading table by page instead of question

%\end{minipage}
\newpage % End of cover page

%%%%%%%%%%%%%%%%%%%%%%%%%%%%%%%%%%%%%%%%%%%%%%%%%%%%%%%%%%%%%%%%%%%%%%%%%%%%%%%%%%%%%
%
% See http://www-math.mit.edu/~psh/#ExamCls for full documentation, but the questions
% below give an idea of how to write questions [with parts] and have the points
% tracked automatically on the cover page.
%
%
%%%%%%%%%%%%%%%%%%%%%%%%%%%%%%%%%%%%%%%%%%%%%%%%%%%%%%%%%%%%%%%%%%%%%%%%%%%%%%%%%%%%%

\begin{questions}

\section{Planning}

% Basic question
\addpoints
% \question[10] Differentiate $f(x)=x^2$ with respect to $x$.
\question[1]
A database will be used primarily to identify sales patterns
for products sold within the last three years and to summarize
sales by region, on a quarterly basis. Which type of system is 
needed?
\begin{choices}
\choice Analytical
\choice DB2 pureScale
\choice Data warehouse
\choice Online transaction processing (OLTP)
\end{choices}

\question[1]
Which product can be used to tune performance for a single query?
\begin{choices}
\choice IBM Data Studio
\choice IBM Control Center
\choice IBM Data Administrator
\choice IBM Workload Manager
\end{choices}

\question[1]
Which two DB2 products are suitable for very large data warehouse applications? (Choose two.)
\begin{choices}
\choice DB2 for i
\choice DB2 for AIX
\choice DB2 for z/OS
\choice DB2 pureScale
\choice DB2 Express-C
\end{choices}

\question[1]
What is the DB2 Workload Manager (WLM) used for?
\begin{choices}
\choice To identify, diagnose, solve, and prevent performance problems in DB2 products and associated applications.
\choice To customize execution environments for the purpose of controlling system resources so that one
		department or service class does not overwhelm the system.
\choice To respond to significant changes in a database's workload by dynamically distributing available memory resources
		among several different database memory consumers.
\choice To improve the performance of applications that require frequent, but relatively transient, simultaneous user
		connections by allocating host database resources only for the duration of an SQL transaction.
\end{choices}

\question[1]
Which of the following is NOT a characteristic of a data warehouse?
\begin{choices}
\choice Sub-second response time
\choice Voluminous historical data
\choice Heterogeneous data sources
\choice Summarized queries that perform aggregations and joins
\end{choices}

\newpage
\addpoints
\question[1]
Which statement about the DB2 pureScale feature is NOT true?
\begin{choices}
\choice The DB2 pureScale feature provides a database cluster solution for nonmainframe platforms.
\choice The DB2 pureScale feature is only available as part of DB2 Advanced Enterprise Server Edition.
\choice The DB2 pureScale feature can only work with the General Parallel File System (GPFS) file system.
\choice The DB2 pureScale feature is best suited for online transaction processing (OLTP) workloads.
\end{choices}

\question[1]
Which two statements about large object (LOB) locators are true? (Choose two.)
\begin{choices}
\choice A LOB locator represents a value for a LOB resource that is stored in a database.
\choice A LOB locator is a simple token value that is used to refer to a much bigger LOB value.
\choice A LOB locator is a special data type that is used to store LOB data in external binary files.
\choice A LOB locator represents a value for a LOB resource that is stored in an external binary file.
\choice A LOB locator is a mechanism that acts similar to an index in the way that is organizes LOB values
		so they can be quickly located in response to a query.
\end{choices}

\question[1]
Which type of database workload typically involves making changes to a small number of records within
a single transaction?
\begin{choices}
\choice Decision support
\choice Data warehousing
\choice Online analytical processing (OLAP)
\choice Online transaction processing (OLTP)
\end{choices}

\question[1]
Which of the following is NOT a characteristic of an OLTP database?
\begin{choices}
\choice Current data
\choice Frequent updates
\choice Granular transactions
\choice Optimized for queries
\end{choices}

\question[1]
Which two platforms support DB2 10.1 pureScale environments? (Choose two.)
\begin{choices}
\choice IBM mainframes running z/OS
\choice IBM p Series servers running AIX
\choice IBM p Series servers running Linux
\choice IBM x Series servers running Linux
\choice IBM x Series servers running a supported version of Windows
\end{choices}

\newpage
\addpoints
\question[1]
Which tool can analyze and provide recommendations for tuning individual queries?
\begin{choices}
\choice IBM InfoSphere Data Architect
\choice IBM InfoSphere Optim Query Tuner
\choice IBM InfoSphere Optim pureQuery Runtime
\choice IBM InfoSphere Optim Performance Manager Extended Edition
\end{choices}

\question[1]
Which SQL statement will create a table named EMPLOYEE that can be used to store XML data?
\begin{choices}
\choice CREATE TABLE employee (empid INT, resume XML)
\choice CREATE TABLE employee (empid INT, resume XML(2000))
\choice CREATE TABLE employee (empid INT, resume CLOB AS XML)
\choice CREATE TABLE employee (empid INT, resume CLOB USING XML)
\end{choices}

\question[1]
What DB2 product provides a complete data warehousing solution that contains components
that facilitate data warehouse construction and administration?
\begin{choices}
\choice DB2 pureScale Feature
\choice DB2 Workload Manager
\choice IBM InfoSphere Warehouse
\choice Database Partitioning Feature
\end{choices}

\question[1]
Which statement about IBM Data Studio is NOT true?
\begin{choices}
\choice The IBM Data Studio administration client can be installed on servers running Red Hat Linux, SUSE Linux,
		Windows, and AIX.
\choice IBM Data Studio replaces the DB2 Control Center as the standard GUI interface for DB2 database administration
		and application development.
\choice IBM Data Studio is an Eclipsed-based, integrated development environment (IDE) that can be used to perform
		instance and database administration.
\choice IBM Data Studio allows users to connect to a DB2 database using a wizard; however, users are required to provide
		login credentials before a connection will be established.
\end{choices}

\question[1]
Which statement about inline large objects (LOBs) is NOT true?
\begin{choices} 
\choice When a table contains columns with inline LOBs, fewer rows can fit on a page.
\choice Inline LOBs are created by appending the INLINE LENGTH clause to a LOB column's definition.
\choice Because DML operations against inline LOBs are never logged, their use can reduce logging overhead.
\choice Inline LOBs improve query performance by storing LOB data in the same data pages as the rest of a table's rows,
		rather than in a separate LOB storage object.
\end{choices}

\newpage
\addpoints
%\fullwidth{\Large \textbf{Security}}
\section{Security}

\question[1]
A user named USER1 has been granted \texttt{DATAACCESS} authority for a database named PAYROLL. What is user USER1 allowed to do?
\begin{choices}
	\choice Implicitly create a new schema in the PAYROLL database.
	\choice Grant and revoke priviledges on objects that reside in the PAYROLL database.
	\choice Retrieve and change data stored in user tables, views. and materialized query tables.
	\choice Create database objects, issue database-specific DB2 commands, and run DB2 utilities that 
	do not change data.
\end{choices}

\question[1]
Which attribute is NOT needed to define a trusted context?
\begin{choices}
\choice A system authorization ID
\choice A data stream encryption value
\choice A system authorization password
\choice The IP address or domain name of an incoming connection
\end{choices}

\question[1]
Which SQL statement will take the ability to run an Embedded SQL application named 
PERF\textunderscore REVIEW that calls a package named CORP.CALC\textunderscore Bonus away from a user named USER1?
\begin{choices}
\choice \texttt{REVOKE EXECUTION ON APPLICATION} perf\textunderscore review \texttt{FROM} user1
\choice \texttt{REVOKE EXECUTION ON PACKAGE} corp.calc\textunderscore bonus \texttt{FROM} user1
\choice \texttt{REVOKE EXECUTION ON APPLICATION} perf\textunderscore review \texttt{PACKAGE} corp.calc
\textunderscore bonus \texttt{FROM} user1
\choice \texttt{REVOKE EXECUTION ON APPLICATION} perf\textunderscore review \texttt{USING PACKAGE} corp.calc\textunderscore bonus \texttt{FROM} user1
\end{choices}


\question[1]
If a user is given \texttt{SELECT} privilege on a table named EMPLOYEES, which two actions are they
allowed to perform? (Choose two.)
\begin{choices}
\choice Add data to the EMPLOYEE table.
\choice Create a view on the EMPLOYEE table.
\choice Retrieve data from the EMPLOYEE table.
\choice Create an index for the EMPLOYEE table.
\choice Change the definition for the EMPLOYEE table.
\end{choices}

\question[1]
Which SQL statement will allow a user named USER1 to both remove records from a table named
SALES and give the ability to remove records from the SALES table to others?
\begin{choices}
\choice \texttt{GRANT DELETE ON TABLE} sales \texttt{TO} user1 \texttt{WITH GRANT OPTION}
\choice \texttt{GRANT REMOVE ON TABLE} sales \texttt{TO} user1 \texttt{WITH GRANT OPTION}
\choice \texttt{GRANT DELETE ON TABLE} sales \texttt{TO} user1 \texttt{WITH GRANT PRIVILEGES}
\choice \texttt{GRANT REMOVE ON TABLE} sales \texttt{TO} user1 \texttt{WITH GRANT PRIVILEGES}
\end{choices}

\newpage

\question[1]
If a user is granted the \texttt{BIND} privilege, what are they allowed to do?
\begin{choices}
\choice Create a new package.
\choice Bind or rebind (recreate) a specific package.
\choice Register user-defined functions (UDFs) and procedures.
\choice Associate user-defined functions (UDFs) and procedures with specific database objects.
\end{choices}

\question[1]
Which statement about \texttt{Security Administrator (SECADM)} authority is true?
\begin{choices}
\choice Users with \texttt{SECADM} authority are not allowed to access data stored in system 
catalog tables and views.
\choice Only users with \texttt{SECADM} authority are allowed to grant and revoke \texttt{SECADM}
authority to/from others.
\choice When a user with \texttt{SECADM} authority creates a database, that user is automatically granted
\texttt{DBADM} authority for that database.
\choice With DB2 for z/OS, \texttt{SYSADM} authority and \texttt{SECADM} authority are combined under
\texttt{SYSADM} authority and cannot be separated.
\end{choices}

\question[1]
Which statement about trusted context is true?
\begin{choices}
\choice Trusted context objects can only be defined by someone with \texttt{SYSADM} or \texttt{SECADM}
authority.
\choice An authorization ID, IP address, encryption value, and authentication type must be identified
before a trusted context can be defined.
\choice After a trusted connection is established, if a switch request is made with an authorization ID
that is not allowed on the connection, the connection is placed in the "Unconnected" state.
\choice If a trusted context is assigned to a role, any authorization ID that uses the trusted context
will acquire the authorities and privileges that have been assigned to the role; any authorities or
privileges that have been granted to the authorization ID are ignored.
\end{choices}

\question[1]
If a user has \texttt{ACCESSCTRL} authority, which two authorities and/or privileges are they allowed
to grant to others? (Choose two.)
\begin{choices}
\choice \texttt{SYSADM}
\choice \texttt{SECADM}
\choice \texttt{EXECUTE}
\choice \texttt{CREATETAB}
\choice \texttt{ACCESSCTRL}
\end{choices}

\newpage

\question[1]
Which of the following is used to group a collection of privileges together so that they can be
simultaneously granted to and revoked from multiple users?
\begin{choices}
\choice Role
\choice Catalog
\choice Function
\choice Collection
\end{choices}

\question[1]
Which method for restricting data access relies on the server or the local DB2 subsystem to prevent
unauthorized users from accessing data stored in a database?
\begin{choices}
\choice Privileges
\choice Authentication
\choice Label-based access control
\choice Row and column access control
\end{choices}

\question[1]
When is an SQL search condition used to limit access to data in a table?
\begin{choices}
\choice When \texttt{mandatory access control (MAC)} is used to protect the table.
\choice When \texttt{label-based access control (LBAC)} is used to protect the table.
\choice When \texttt{discretionary access control (DAC)} is used to protect the table.
\choice When \texttt{row and column access control (RCAC)} is used to protect the table.
\end{choices}

\question[1]
Which SQL statement will give user USER1 the ability to create tables in a table space named USERSPACE2?
\begin{choices}
\choice \texttt{GRANT USE OF TABLESPACE} userspace2 \texttt{TO} user1
\choice \texttt{GRANT ALTER ON TABLESPACE} userspace2 \texttt{TO} user1
\choice \texttt{GRANT USAGE OF TABLESPACE} userspace2 \texttt{TO} user1
\choice \texttt{GRANT CREATETAB ON TABLESPACE} userspace2 \texttt{TO} user1
\end{choices}

\question[1]
Which SQL statement will give user USER1 the ability to assign a comment to a table named MYTABLE?
\begin{choices}
\choice \texttt{GRANT ALTER ON TABLE} mytable \texttt{TO} user1
\choice \texttt{GRANT USAGE ON TABLE} mytable \texttt{TO} user1
\choice \texttt{GRANT INSERT ON TABLE} mytable \texttt{TO} user1
\choice \texttt{GRANT UPDATE ON TABLE} mytable \texttt{TO} user1
\end{choices}

\question[1]
Which privileges are needed to invoke an SQL stored procedure that queries a table?
\begin{choices}
\choice \texttt{CALL} privilege on the procedure; \texttt{SELECT} privilege on the table.
\choice \texttt{EXECUTE} privilege on the procedure; \texttt{SELECT} privilege on the table.
\choice \texttt{CALL} privilege on the procedure; \texttt{REFERENCES} privilege on the table.
\choice \texttt{EXECUTE} privilege on the procedure; \texttt{REFERENCES} privilege on the table.
\end{choices}

\newpage

\question[1]
Which privileges allows a user to use the \texttt{PREVIOUS VALUE} and \texttt{NEXT VALUE} sequence 
expressions?
\begin{choices}
\choice \texttt{USE}
\choice \texttt{ALTER}
\choice \texttt{USAGE}
\choice \texttt{EXECUTE}
\end{choices}

\question[1]
A table named CUSTOMER was created as follows:
\begin{verbatim}
CREATE TABLE customer
(cust_id   INTEGER NOT NULL PRIMARY KEY,
 f_name    VARCHAR(30),
 l_name    VARCHAR(40),
 cc_number NUMERIC(16,0) NOT NULL)
\end{verbatim}
Which two actions will prevent unauthorized users from accessing credit card number (CC\textunderscore 
NUMBER) information? (Choose two.)
\begin{choices}
\choice Assign the CC\textunderscore NUMBER column to a restricted role that only authorized users
are allowed to access.
\choice Only grant \texttt{ACCESSCTRL} authority for the CC\textunderscore NUMBER column to users who
need to access credit card number information.
\choice Alter the table definition so that CC\textunderscore NUMBER data is stored in a separate schema
that only authorized users are allowed to access.
\choice Create a view for the CUSTOMER table that does not contain the CC\textunderscore NUMBER column
and require unauthorized users to use the view.
\choice Create a column mask for the CC\textunderscore NUMBER column with the ENABLE option specified and 
alter the CUSTOMER table to activate column access control.
\end{choices}

\question[1]
Which authority is needed to create and drop databases?
\begin{choices}
\choice \texttt{DBADM}
\choice \texttt{DBCTRL}
\choice \texttt{SYSCTRL}
\choice \texttt{SYSMAINT}
\end{choices}

\question[1]
Which statement regarding \texttt{label-based access control (LBAC)} is true?
\begin{choices}
\choice Two types of security label components are supported: array and tree.
\choice Every LBAC-protected table must have only one security policy associated with it.
\choice To configure a table for row-level LBAC protect, you must include the \texttt{SECURED WITH}
clause with each column's definition.
\choice To configure a table for column-level LBAC protection, you must include a column with the
\texttt{DB2SECURITYLABEL} data type in the table's definition.
\end{choices}

\newpage

\question[1]
Which method for restricting data access relies on an SQL CASE expression to control the conditions
under which a user can access for a column?
\begin{choices}
\choice Authority
\choice Authentication
\choice Label-based access control
\choice Row and column access control
\end{choices}

\question[1]
Which two statements about Row and column Access Control (RCAC) are valid? (Choose two.)
\begin{choices}
\choice A column mask's access control rule is defined by an SQL search condition.
\choice A column mask's access control rule is defined by an SQL CASE expression.
\choice A row permission's access control rule is defined by an SQL search condition.
\choice A row permission's access control rule is defined by an SQL CASE expression.
\choice A column mask's access control rule is defined by a \texttt{SECURED WITH} clause of 
a \texttt{CREATE TABLE} or \texttt{ALTER TABLE} statement.
\end{choices}

\question[1]
Which privilege is needed to invoke a stored procedure?
\begin{choices}
\choice \texttt{USE}
\choice \texttt{CALL}
\choice \texttt{USAGE}
\choice \texttt{EECUTE}
\end{choices}

\newpage
%\fullwidth{\Large \textbf{Working with Databases and Database Objects}}
\section{Working with Databases and Database Objects}
\question[1]
Which statement about views is NOT true?
\begin{choices}
\choice A view can be defined as being updatable or read-only.
\choice Views obtain their data from the table(s) or view(s) they are based on.
\choice A view can be used to limit a user's ability to retrieve data from a table
\choice The SQL statement provided as part of a view's definition determines what data is 
presented when the view is referenced.
\end{choices}

\question[1]
If the following SQL statement is executed: 
\begin{verbatim}
CREATE DISTINCT TYPE pound_sterling AS DECIMAL (9,2) WITH COMPARISONS
\end{verbatim}
Which event will NOT happen?
\begin{choices}
\choice A user-defined data type that can be used to store numerical data as British currency will be
created.
\choice Six comparison functions will be created so that POUND\_STERLING values can be compared to each
other.
\choice Two casting functions will be created so that POUND\_STERLING values can be converted to DECIMAL
values, and vice versa.
\choice A compatibility function will be created so all of DB2's built-in functions that accept DECIMAL
values as input can be used with POUND\_STERLING data.
\end{choices}

\question[1]
If the following SQL statements are executed:
\begin{verbatim}
CREATE TABLE sales(
	order_num      INTEGER NOT NULL,
	customer_name  VARCHAR(50),
	amount_due     DECIMAL(6,2));
CREATE UNIQUE INDEX idx_ordernum ON sales(order_num);
\end{verbatim}
Which two statements are true? (Choose two.)
\begin{choices}
\choice Every ORDER\_NUM value must be unique.
\choice Duplicate ORDER\_NUM values are allowed.
\choice No other indexes can be created for the SALES table.
\choice A query will return rows from the SALES table in no specific order.
\choice Index IDX\_ORDERNUM will serve as the primary key for the SALES table.
\end{choices}

\question[1]
What is the minimum product that is needed to give applications running on personal computers the 
ability to work with DB2 databases that reside on System z platforms, without using a gateway?
\begin{choices}
\choice DB2 Connect Personal Edition
\choice DB2 Connect Enterprise Edition
\choice IBM DB2 Connect Unlimited Advanced Edition for System z
\choice IBM DB2 Connect Unlimited Advanced Edition for System i
\end{choices}

\newpage
\question[1]
Which action does NOT need to be performed to complete the definition of an application-period temporal
table?
\begin{choices}
\choice A business-time-begin column must be created for the table.
\choice A business-time-end column must be created for the table.
\choice A BUSINESS\_TIME period must be specified in a CREATE or ALTER of the table.
\choice A unique index must be created that prevents overlapping of the BUSINESS\_TIME period of the 
table.
\end{choices}

\question[1]
What are buffer pools used for?
\begin{choices}
\choice To cache table and index data as it is read from disk.
\choice To keep track of changes that are made to a database as they occur.
\choice To control the amount of processor resources that SQL statements can consume.
\choice To provide a layer of indirection between a data object and the storage where that object's data
resides.
\end{choices}

\question[1]
Which statement regarding distributed requests is NOT true?
\begin{choices}
\choice To implement distributed request functionality, all you need is a federated database and one
or more remote data sources.
\choice Distributed request functionality allows a UNION operation to be performed between a DB2 table and
an Oracle view.
\choice Distributed request functionality allows SQL operations to reference two or more databases or 
relational database management systems in a single statement.
\choice DB2 Connect provides the ability to perform distributed requests across members of the DB2 Family,
as well as across other relational database management systems.
\end{choices}

\question[1]
Which statement about indexes is NOT true?
\begin{choices}
\choice An index can be used to enforce the uniqueness of records in a table.
\choice Indexes provide a fast, efficient method for locating specific rows in a table.
\choice When an index is created, metadata for the index is stored in the system catalog.
\choice Indexes automatically provide both a logical and physical ordering of the rows in a table.
\end{choices}

\question[1]
What are Materialized Query Tables (MQTs) used for?
\begin{choices}
\choice To physically cluster data on more than one dimension, simultaneously.
\choice To improve the execution performance of qualified SELECT statements.
\choice To hold nonpresistent data temporarily, on behalf of a single application.
\choice To track effective dates for data that is subject to changing business conditions.
\end{choices}

\newpage
\question[1]
Which two actions must be performed to track changes made to a system-period temporal table over time?
(Choose two.)
\begin{choices}
\choice A history table must be created with columns that are identical to those of the system-period
temporal table.
\choice The system-period temporal table must be altered using the ADD VERSIONING clause to relate it to
a history table.
\choice A primary key must be defined for the system-period temporal table that prevents overlapping
of SYSTEM\_TIME periods.
\choice A unique index must be defined on the transaction-start-id column of both the system-period 
temporal table and its associated history table.
\choice The system-period temporal table must be altered to add system-time-begin, system-time-end, 
transaction-start-id, and transaction-end-id columns.
\end{choices}

\question[1]
Which database object can be used to automatically generate a numeric value that is not tied to any
specific column or table?
\begin{choices}
\choice Alias
\choice Schema
\choice Package
\choice Sequence
\end{choices}

\question[1]
Which column is NOT required as part of the table definition for a system-period temporal table?
\begin{choices}
\choice A row-begin column with a TIMESTAMP(12) data type
\choice A row-end column with a TIMESTAMP(12) data type
\choice A transaction-start-id column with a TIMESTAMP(12) data type
\choice A transaction-stop-id column with a TIMESTAMP(12) data type
\end{choices}

\question[1]
Which object can NOT be enabled for compression?
\begin{choices}
\choice Views
\choice Indexes
\choice Base tables
\choice Temporary tables
\end{choices}

\question[1]
What is a schema used for?
\begin{choices}
\choice To provide an alternate name for a table or view.
\choice To provide a logical grouping of database objects.
\choice To generate a series of numbers, in ascending or descending order.
\choice To provide an alternative way of describing data stored in one or more tables.
\end{choices}

\question[1]
Which view definition type is NOT supported?
\begin{choices}
\choice Insertable
\choice Updatable
\choice Read-only
\choice Write-only
\end{choices}

\newpage
\question[1]
When should an application-period temporal table be used?
\begin{choices}
\choice When you want to keep track of historical versions of a table's rows.
\choice When you want to define specific time periods in which data is valid.
\choice When you want to cluster data according to the time in which rows are inserted.
\choice When you want to cluster data on more than one key or dimension, simultaneously.
\end{choices}

\question[1]
Which statement about buffer pools is NOT true?
\begin{choices}
\choice Every table space must have a buffer pool assigned to it.
\choice One buffer pool is created automatically as part of the database creation process.
\choice Dirty pages are automatically removed from a buffer pool when they are written to storage.
\choice Once a page has been copied to a buffer pool, it remains there until the space it occupies is 
needed.
\end{choices}

\question[1]
Which DB2 object can a view NOT be derived from?
\begin{choices}
\choice Alias
\choice View
\choice Table
\choice Procedure
\end{choices}

\question[1]
Which two expressions can be used with a sequence? (Choose two.)
\begin{choices}
\choice NEXT VALUE
\choice PRIOR VALUE
\choice CURRENT VALUE
\choice PREVIOUS VALUE
\choice SUBSEQUENT VALUE
\end{choices}

\question[1]
Which object is a distinct data type defined into?
\begin{choices}
\choice Schema
\choice Package
\choice Database
\choice Table space
\end{choices}

\question[1]
Which two objects can NOT be created in DB2? (Choose two.)
\begin{choices}
\choice Plan
\choice Trigger
\choice Scheme
\choice Function
\choice Sequence
\end{choices}

\newpage
\question[1]
Which statement about Type 2 connections is true?
\begin{choices}
\choice Type 2 connections cannot be used with DB2 for z/OS
\choice Type 2 connections are used by default with DB2 for Linux, Unix, and Windows.
\choice Type 2 connections allow applications to be connected to only one database at a time.
\choice Type 2 connections allow applications to connect to and work with multiple DB2 databases 
simultaneously.
\end{choices}

\question[1]
Which two statements about bitemporal tables are valid? (Choose two.)
\begin{choices}
\choice Bitemporal tables are system tables and can only be queried by the table owner.
\choice When data in a bitemporal table is updated, a row is added to its associated history table.
\choice Creating a bitemporal table is similar to creating a base table except users must define a 
SYSTEM\_TIME\_PERIOD column.
\choice When querying a bitemporal table, you have the option of providing a system time-period 
specification, a business time-period specification, or both.
\choice Bitemporal tables must contain bitemporal-time-begin, bitemporal-time-end, and transaction-start-i
id columns, along with SYSTEM\_TIME and BUSINESS\_TIME periods.
\end{choices}

\question[1]
Which programming interface is widely used for database access because it allows applications to run, 
unchanged, on most hardware platforms?
\begin{choices}
\choice ODBC
\choice SQLJ
\choice JDBC
\choice OLE DB
\end{choices}

\question[1]
Which two types of temporal tables can be used to store time-sensitive data? (Choose two.)
\begin{choices}
\choice Bitemporal
\choice Time-period
\choice System-period
\choice Business-period
\choice Application-period
\end{choices}

\question[1]
In which of the following scenarios would a stored procedure be beneficial?
\begin{choices}
\choice An application running on a remote client needs to track every modification made to a 
table that contains sensitive data.
\choice An application running on a remote client needs to be able to convert degress Celsius to degrees
Fahrenheit and vice versa.
\choice An application running on a remote client needs to ensure that every new employee that joins the 
company is assigned a unique, sequential employee number.
\choice An application running on a remote client needs to collect input values from a user, 
perform a calculation using the values provided, and store the input data, along with the calculation
results, in a base table.
\end{choices}

\newpage
\question[1]
Given the following SQL statement:
\begin{verbatim}
CREATE ALIAS emp_info FOR employees
\end{verbatim}
Which two objects can the name EMPLOYEES refer to? (Choose two.)
\begin{choices}
\choice A view
\choice An alias
\choice An index
\choice A sequence
\choice A procedure
\end{choices}

\question[1]
Which operation can NOT be performed by executing an ALTER SEQUENCE statement?
\begin{choices}
\choice Change a sequence's data type.
\choice Change whether a sequence cycles.
\choice Establish new minimum and maximum sequence values.
\choice Change the number of sequence numbers that are cached.
\end{choices}

\question[1]
Which object must exist before an index can be created?
\begin{choices}
\choice View
\choice Table
\choice Schema
\choice Sequence
\end{choices}

\question[1]
If the following SQL statement is executed:
\begin{verbatim}
CREATE DATABASE payroll
\end{verbatim}
Which two statements are true? (Choose two.)
\begin{choices}
\choice The PAYROLL database will have a page size of 4KB.
\choice The PAYROLL database will have a page size of 8KB.
\choice The PAYROLL database will be an automatic storage database.
\choice The PAYROLL database will not be an automatic storage database.
\choice The PAYROLL database will be assigned the comment "PAYROLL DATABASE."
\end{choices}

\newpage
%\fullwidth{\Large \textbf{Working with DB2 Data Using SQL}}
\section{Working with DB2 Data Using SQL}
\question[1]
If the following result set is desired:
\begin{verbatim}
STATE          REGION          AVG_INCOME
------------- --------------- -------------
MARYLAND      MID-ATLANTIC     86056.00
NEW JERSEY    NORTHEAST        85005.00
CONNECTICUT   NORTHEAST        84558.00
MASSACHUSETTS NORTHEAST        82009.00
ALASKA        PACIFIC-ALASKA   79617.00
\end{verbatim}
Which SQL statement must be executed?
\begin{choices}
\choice \begin{verbatim}
		SELECT state, region, avg_income
		  FROM census_data
		  ORDER BY 3
		  FETCH FIRST 5 ROWS
		\end{verbatim}
\choice \begin{verbatim}
		SELECT state, region, avg_income
		  FROM census_data
		  ORDER BY 3
		  FETCH FIRST 5 ROWS ONLY
	    \end{verbatim}
\choice \begin{verbatim}
		SELECT state, region, avg_income
		  FROM census_data
		  ORDER BY 3 DESC
		  FETCH FIRST 5 ROWS
		\end{verbatim}
\choice \begin{verbatim}
		SELECT state, region, avg_income
		  FROM census_data
		  ORDER BY 3 DESC
		  FETCH FIRST 5 ROWS ONLY
        \end{verbatim}
\end{choices}

\question[1]
Which type of join will usually produce the smallest result set?
\begin{choices}
\choice INNER JOIN
\choice LEFT OUTER JOIN
\choice RIGHT OUTER JOIN
\choice FULL OUTER JOIN
\end{choices}

\question[1]
Which statement about SQL subqueries is NOT true?
\begin{choices}
\choice A subquery can be used with an UPDATE statement to supply values for one or more columns 
that are to be updated.
\choice A subquery can be used with an INSERT statement to retrieve values from one base table or view
and copy them to another.
\choice If a subquery is used with a DELETE statement and the result set produced is empty, every record
will be deleted from the table specified.
\choice If a subquery is used with an UPDATE statement and the result set produced contains multiple rows,
the operation will fail and an error will be generated.
\end{choices}

\newpage
\question[1]
Which statement about savepoints is NOT true?
\begin{choices}
\choice You can use as many savepoints as you desire within a single unit of work, provided you do not
nest them.
\choice Savepoints provide a way to break the work being done by a single large transaction into one or
more smaller subsets.
\choice The COMMIT FROM SAVEPOINT statement is used to commit a subset of database changes that have
been made within a unit of work.
\choice The ROLLBACK TO SAVEPOINT statement is used to back out a subset of database changes that have
been made within a unit of work.
\end{choices}

\question[1]
Which two statements about UPDATE processing are true? (Choose two.)
\begin{choices}
\choice A positioned UPDATE is used to modify one or more rows, and a searched UPDATE is used to 
modify exactly one row.
\choice A searched UPDATE is used to modify one or more rows, and a positioned UPDATE is used to
modify exactly one row.
\choice When the UPDATE statement modifies parent key columns, the values of corresponding foreign key
columns are modified as well.
\choice The UPDATE statement can be used to remove data from specified columns in the rows of a table,
provided those columns are not nullable.
\choice The UPDATE statement can be used to modify the values of specified columns in the rows
of a table, view, or underlying table(s) of a specified fullselect.
\end{choices} 

\question[1]
A table named SALES has two columns: SALES\_AMT and REGION\_CD. Which SQL statement will return the number
of sales in each region, ordered by number of sales made?
\begin{choices}
\choice \begin{verbatim}
		SELECT sales_amt, COUNT(*)
		  FROM sales
		  ORDER BY 2
	 	\end{verbatim}
\choice \begin{verbatim}
		SELECT sales_amt, COUNT(*)
		  FROM sales
		  GROUP BY sales_amt
		  ORDER BY 1
		\end{verbatim}
\choice \begin{verbatim}
		SELECT region_cd, COUNT(*)
		  FROM sales
		  GROUP BY region_cd
		  ORDER BY COUNT(*)
		\end{verbatim}
\choice \begin{verbatim}
		SELECT region_cd, COUNT(*)
		  FROM sales
		  GROUP BY sales_amt
		  ORDER BY COUNT(*)
		\end{verbatim}
\end{choices}

\newpage
\question[1]
A user wants to retrieve records from a table named SALES that satisfy at least one of the following
criteria:
	\begin{itemize}
	\item The sales date (SALESDATE) is after June 1, 2012, and the sales amount (AMT) is greater
	than \$40.00.
	\item The sales was made in the hardware department. 
	\end{itemize}
Which SQL statement will accomplish this?
\begin{choices}
\choice \begin{verbatim}
		SELECT * FROM sales
		  WHERE (salesdate > '2012-06-01' OR (amt > 40
		   AND (dept = 'Hardware')
		\end{verbatim}
\choice \begin{verbatim}
		SELECT * FROM sales
		  WHERE (salesdate > '2012-06-01') OR (amt > 40)
		    OR (dept = 'Hardware')
		\end{verbatim}
\choice \begin{verbatim}
 		SELECT * FROM sales
 		  WHERE (salesdate > '2012-06-01' AND amt > 40
 		   AND (dept = 'Hardware')
		\end{verbatim}
\choice \begin{verbatim}
		SELECT * FROM sales
		  WHERE (salesdate > '2012-06-01' AND amt > 40)
		   OR (dept = 'Hardware')
		\end{verbatim}
\end{choices}

\question[1]
Which two statements about INSERT operations are true? (Choose two.)
\begin{choices}
\choice The INSERT statement can be used to insert rows into a table, view, or table function.
\choice Inserted values must satisfy the conditions of any check constraints that have been defined on the
table specified.
\choice The values provided in the VALUES clause of an INSERT statement are assigned to columns in the 
order in which they appear.
\choice If an INSERT statement omits any column from the inserted row that is defined as NULL or NOT
NULL WITH DEFAULT, the statement will fail.
\choice If the underlying table of a view that is referenced by an INSERT statement has one or more
unique indexes, each row inserted does not have to conform to the constraints imposed by those indexes.
\end{choices}

\question[1]
Which SQL statement should be used to select the minimum and maximum rainfall amounts (RAINFALL), by 
month (MONTH), from a table named WEATHER?
\begin{choices}
\choice \begin{verbatim}
		SELECT month, MIN(rainfall), MAX(rainfall)
		  FROM weather
		  ORDER BY month
		\end{verbatim}
\choice \begin{verbatim}
		SELECT month, MIN(rainfall), MAX(rainfall)
		  FROM weather
		  GROUP BY month
		\end{verbatim}
\choice \begin{verbatim}
		SELECT month, MIN(rainfall), MAX(rainfall)
		  FROM weather
		  GROUP BY month, MIN(rainfall), MAX(rainfall)
		\end{verbatim}
\choice \begin{verbatim}
		SELECT month, MIN(rainfall), MAX(rainfall)
		  FROM weather
		  ORDER BY month, MIN(rainfall), MAX(rainfall)
		\end{verbatim}
\end{choices}

\newpage
\question[1]
An SQL function named REGIONAL\_SALES was created as follows:
	\begin{verbatim}
	CREATE FUNCTION regional_sales()
		RETURNS TABLE (region_id VARCHAR(20),
					   sales_amt DECIMAL(8,2))
		READS SQL DATA
		BEGIN ATOMIC
		RETURN
			SELECT region, amt
			FROM sales
			ORDER BY region;
		END
	\end{verbatim}
Which two statements demonstrate the proper way to use this function in a query? (Choose two.)
\begin{choices}
\choice \texttt{SELECT * FROM regional\_sales()}
\choice \texttt{SELECT regional\_sales (region\_id, sales\_amt)}
\choice \texttt{SELECT region\_id, sales\_amt FROM regional\_sales()}
\choice \texttt{SELECT * FROM TABLE (regional\_sales()) AS results}
\choice \texttt{SELECT region\_id, sales\_amt FROM TABLE(region\_sales()) AS results}
\end{choices}

\question[1]
If the following SQL statement is executed:
\begin{verbatim}
SELECT dept, AVG(salary)
 FROM employee
 GROUP BY dept
 ORDER BY 2
\end{verbatim}
What will be the results?
\begin{choices}
\choice The department number and average salary for all employees will be retrieved from a table named
EMPLOYEE, and the results will be arranged in descending order, by department.
\choice The department number and average salary for all departments will be retrieved from a table named
EMPLOYEE, and the results will be arranged in ascending order, by department.
\choice The department number and average salary for all employees will be retrieved from a table named
EMPLOYEE, and the results will be arranged in ascending order, by average departmental salary.
\choice The department number and average salary for all departments will be retrieved from a table named
EMPLOYEE, and the results will be arranged in descending order, by average departmental salary.
\end{choices}

\newpage
\question[1]
If a table named SALES contains information about invoices that do not have a negative balance, which 
two SQL statements can be used to retrieve invoice numbers for invoices that are for less than \$25,000.00
? (Choose two.)
\begin{choices}
\choice \texttt{SELECT invoice\_num FROM sales WHERE amt < 25000}
\choice \texttt{SELECT invoice\_num FROM sales WHERE amt < 25000}
\choice \texttt{SELECT invoice\_num FROM sales WHERE amt LESS THAN 25000}
\choice \texttt{SELECT invoice\_num FROM sales WHERE amt BETWEEN 0 AND 25000}
\choice \texttt{SELECT invoice\_num FROM sales WHERE amt BETWEEN 0 AND 25000}
\end{choices}

\question[1]
An SQL function designed to convert miles to kilometers was created as follows:
\begin{verbatim}
CREATE FUNCTION mi_to_km (IN miles FLOAT)
  RETURN FLOAT
  LANGUAGE SQL
  SPECIFIC convert_mtok
  READS SQL DATA
  RETURN FLOAT (miles * 1.60934)
\end{verbatim}
How can this function be used to convert miles (MILES) values stored in a table named DISTANCES?
\begin{choices}
\choice CALL mi\_to\_km (distances.miles)
\choice CALL convert\_mtok (distances.miles)
\choice SELECT mi\_to\_km (miles) FROM distances
\choice SELECT convert\_mtok (miles) FROM distances
\end{choices}

\question[1]
Which two statements about system-period temporal tables are true? (Choose two.)
\begin{choices}
\choice They store user-based period information.
\choice They do not require a separate history table.
\choice They store system-based historical information.
\choice They can be queried without a time-period specification.
\choice They manage data based on time criteria specified by users or applications.
\end{choices} 

\question[1]
A table named PARTS contains a record of every part that has been manufactured by a company. A user
wishes to see the total number of parts that have been made by each craftsman employed at the company.
Which SQL statement will produce the desired results?
\begin{choices}
\choice \texttt{SELECT name, COUNT(*) AS parts\_made FROM parts}
\choice \texttt{SELECT name, COUNT(*) AS parts\_made FROM parts GROUP BY name}
\choice \texttt{SELECT name, COUNT(DISTINCT name) AS parts\_made FROM parts}
\choice \texttt{SELEC DISTINCT name, COUNT(*) AS parts\_made FROM parts GROUP BY parts\_made}
\end{choices}

\newpage
\question[1]
Given an EMPLOYEES table and a DEPARTMENT table, a user wants to produce a list of all departments and
employees who work in them, including departments that no employees have been assigned to. Which SQL
statement will produce the desired list?
\begin{choices}
\choice \begin{verbatim}
		SELECT employees.name, departments.deptname
		FROM employees
		INNER JOIN department ON
		employees.dept = departments.deptno
		\end{verbatim} 
\choice	\begin{verbatim}
		SELECT employees.name, departments.deptname
		FROM employees
		INNER JOIN department ON
		departments.deptno = employees.dept
		\end{verbatim}
\choice \begin{verbatim}
		SELECT employees.name, departments.deptname
		FROM employees
		LEFT OUTER JOIN departments ON
		employees.dept = departments.deptno
		\end{verbatim}
\choice \begin{verbatim}
		SELECT employees.name, departments.deptname,
		FROM employees
		RIGHT OUTER JOIN departments ON
		employees.dept = departments.deptno
		\end{verbatim}
\end{choices}

\question[1]
Which statements are NOT allowed in the body of an SQL scalar user-defined function?
\begin{choices}
\choice CALL statements
\choice COMMIT statements
\choice SQL CASE statements
\choice SQL control statements
\end{choices}

\newpage
\question[1]
A table named TABLE\_A contains 200 rows and a user wants to update the 10 rows in this table with the
lowest values in a column named COL1. Which SQL statement will produce the desired results?
\begin{choices}
\choice \begin{verbatim}
		UPDATE
		  (SELECT * FROM table_a
		    ORDER BY col1 ASC) AS temp
		  SET col2 = 99
		  FETCH FIRST 10 ROWS ONLY
		\end{verbatim}
\choice \begin{verbatim}
		UPDATE
		  (SELECT * FROM table_a
		    ORDER BY col1 DESC) AS temp
		  SET col2=99
		  FETCH FIRST 10 ROWS ONLY
		\end{verbatim}
\choice \begin{verbatim}
		UPDATE
		  (SELECT * FROM table_a
		    ORDER BY col1 ASC
	  		FETCH FIRST 10 ROWS ONLY) AS temp
	  	  SET col2=99
		\end{verbatim}
\choice \begin{verbatim}
		UPDATE
		  (SELECT * FROM table_a
		    ORDER BY col1 DESC
		    FETCH FIRST 10 ROWS ONLY) AS temp
		  SET col2 = 99
		\end{verbatim}
\end{choices}

\question[1]
Which statement best descirbes a transaction?
\begin{choices}
\choice A transaction is a recoverable sequence of operations whose point of consistency is established
only when a savepoint is created.
\choice A transaction is recoverable sequence of operations whose point of consistency can be obtained
by querying the system catalog tables.
\choice A transaction is a recoverable sequence of operations whose point of consistency is established
when a database connection is established or a savepoint is created.
\choice A transaction is recoverable sequence of operations whose point of consistency is established
when an executable SQL statement is processed after a database connection has been established or a
previous transaction has been terminated.
\end{choices}

\question[1]
A table named TABLE\_A contains 200 rows and a user wants to delete
the last 10 rows from this table. Which SQL statement will produce the
desired results?
\begin{choices}
\choice \begin{verbatim}
		DELETE FROM
		 (SELECT * FROM table_a 
		   ORDER BY col1 ASC
		   FETCH FIRST 10 ROWS ONLY) AS result
		\end{verbatim}
\choice \begin{verbatim}
		DELETE FROM
		  (SELECT * FROM table_a
			ORDER BY col1 ASC
			FETCH LAST 10 ROWS ONLY) AS result
		\end{verbatim}
\choice \begin{verbatim}
		DELETE FROM
		 (SELECT * FROM table_a
		   ORDER BY col1 ASC
		   FETCH LAST 10 ROWS ONLY) AS result
		\end{verbatim}
\choice \begin{verbatim}
		DELETE FROM
		 (SELECT * FROM table_a
		   ORDER BY col1 DESC
		   FETCH LAST 10 ROWS ONLY) AS result
		\end{verbatim}
\end{choices}

\question[1]
Which clause could be added to the following SQL statement
\begin{verbatim}
SELECT student_id, enroll_date, gpa
 FROM students
\end{verbatim}
to ensure that only information (STUDENT\_ID, ENROLL\_DATE, and
GPA) for students who started school before 2012 and who have a GPA
that is higher than 3.50 will be retrieved?
\begin{choices}
\choice FOR enroll\_date <'2012-01-01' OR gpa>3.50
\choice FOR enroll\_date <'2012-01-01' AND gpa>3.50
\choice WHERE enroll\_date <'2012-01-01' OR gpa>3.5
\choice	WHERE enroll\_date <'2012-01-01' AND gpa>3.5
\end{choices}

\question[1]
Which statement correctly describes what a native SQL stored procedure is?
\begin{choices}
\choice A procedure whose body is written entirely in SQL or SQL PL.
\choice A procedure that is written in a high-level programming language such as Java or REXX.
\choice A procedure whose body is written entirely in SQL, but that is implemented as an external program.
\choice	A procedure that accesses data using an Object Linking and Embedding, Database (OLE DB) provider.
\end{choices}

\question[1]
Given the following statements:
\begin{verbatim}
CREATE TABLE customer (custid INTEGER, custinfo XML);
INSERT INTO customer VALUES (100,
'<customerinfo>
  <name>ACME Manufacturing</name>
  <addr country="United States">
     <street>25 Elm Street</street>
 <city>Raleigh</city>
 <state>North Carolina</state>
 <zip>27603</zip>
  </addr>
</customerinfo>');
\end{verbatim}
If the following XQuery statement is executed:
\begin{verbatim}
XQUERY
for $info in db2-fn:xmlcolumn('CUSTOMER.CUSTINFO')/customerinfo
return $info/name
\end{verbatim}
What will be the result?
\begin{choices}
\choice \texttt{ACME Manufacturing}
\choice \texttt{<name>ACME Manufacturing</name>}
\choice \texttt{<customerinfo>ACME Manufacturing</customerinfo>}
\choice \texttt{<customerinfo><name>ACME Manufacturing</name></customerinfo>}
\end{choices}

\question[1]
Which two statements about roll back operations are correct? (Choose two.)
\begin{choices}
\choice When a ROLLBACK statement is executed, all locks held by the terminating transaction are released.
\choice When a ROLLBACK TO SAVEPOINT statement is executed, all locks acquired after the most recent
savepoint are released.
\choice When a ROLLBACK statement is executed, all locks acquired for open cursors that were 
declared WITH HOLD are held.
\choice When a ROLLBACK TO SAVEPOINT statement is executed, all locks acquired up to the most recent
savepoint are released.
\choice When a ROLLBACK TO SAVEPOINT statement is executed, a savepoint is not automatically deleted as
part of the rollback operation.
\end{choices}

\question[1]
Which SQL statement illustrates the proper way to perform a positioned update operation on a table named
SALES?
\begin{choices}
\choice \texttt{UPDATE sales SET amt = 102.45}
\choice \texttt{UPDATE sales SET amt = 102.45 WHERE cust\_id='000290'}
\choice \texttt{UPDATE sales SET amt = 102.45 WHERE CURRENT OF cursor1}
\choice \begin{verbatim}
		UPDATE sales SET amt = 102.45
		WHERE ROWID = (SELECT ROWID FROM sales WHERE cust_id='000290')
		\end{verbatim}
\end{choices}

\question[1]
Which two statements about application-period temporal tables are true? (Choose two.)
\begin{choices}
\choice They are useful when one wants to keep user-based time period information.
\choice They consist of explicitly supplied timestamps and a separate associated history table.
\choice They are useful when one wants to keep both user-based and system-based time period information.
\choice They are based on explicitly supplied timestamps that define the time periods during which
data is valid.
\choice They consist of a pair of columns with database-manager maintained values that indicate the 
period when a row is current.
\end{choices}

\newpage
\question[1]
When should the TRUNCATE statement be used?
\begin{choices}
\choice When you want to delete all rows from a table without generating log records.
\choice When you want to delete select rows from a table without generating log records.
\choice When you want to delete all rows from a table and fire any delete triggers that have been 
defined for the table.
\choice When you want to delete select rows from a table and fire any delete triggers that have been
defined for the table.
\end{choices}

\question[1]
Which two commands will terminate the current transaction and start a new transaction boundary?
(Choose two.)
\begin{choices}
\choice COMMIT
\choice REFRESH
\choice RESTART
\choice CONNECT
\choice ROLLBACK
\end{choices}

\question[1]
Which SQL statement should be used to retrieve the minimum and maximum annual temperature (TEMP) for
each major city (CITY), sorted by city, from a table named WEATHER?
\begin{choices}
\choice \begin{verbatim}
		SELECT city, MIN(temp), MAX(temp)
		  FROM weather
		  ORDER BY city
		\end{verbatim}
\choice \begin{verbatim}
		SELECT city, MIN(temp), MAX(temp)
		  FROM weather
		  GROUP BY city
		\end{verbatim}
\choice \begin{verbatim}
		SELECT city, MIN(temp), MAX(temp)
		  FROM weather
		  GROUP BY city
		  ORDER BY city
		\end{verbatim}
\choice \begin{verbatim}
		SELECT city, MIN(temp), MAX(temp)
		  FROM weather
		  GROUP BY MIN(temp), MAX(temp)
		  ORDER BY city
		\end{verbatim}
\end{choices}

\question[1]
What is the XMLTABLE() function typically used for?
\begin{choices}
\choice To convert a well-formed XML document into a table of character string values.
\choice To obtain values from XML documents that are to be inserted into one or more tables.
\choice To parse a character string value and return a table of well-formed XML documents.
\choice To produce a temporary table whose columns are based on the elements found in a well-formed XML
document.
\end{choices}
% Question with parts
% \newpage
% \addpoints
% \question Consider the function $f(x)=x^2$.
% \begin{parts}
% \part[5] Find $f'(x)$ using the limit definition of derivative.
% \vfill
% \part[5] Find the line tangent to the graph of $y=f(x)$ at the point where $x=2$.
% \vfill
% \end{parts}

% If you want the total number of points for a question displayed at the top,
% as well as the number of points for each part, then you must turn off the point-counter
% or they will be double counted.
% \newpage
% \addpoints
% \question[10] Consider the function $f(x)=x^3$.
% \noaddpoints % If you remove this line, the grading table will show 20 points for this problem.
% \begin{parts}
% \part[5] Find $f'(x)$ using the limit definition of derivative.
% \vspace{4.5in}
% \part[5] Find the line tangent to the graph of $y=f(x)$ at the point where $x=2$.
% \end{parts}



\end{questions}
\end{document}